\section{Übersetzungen und Drehzahlen}
\subsubsection{Festlegen der Gänge: }
\begin{align*}
\text{Gang 1: }& Z1/Z2 \text{ und } Z6/Z7 \\
\text{Gang 2: }& Z1/Z2 \text{ und } Z4/Z5 \\
\text{Gang 3: }& Z1/Z3 \text{ und } Z6/Z7 \\
\text{Gang 2: }& Z1/Z3 \text{ und } Z4/Z5 \\
\end{align*}
\subsubsection{Gegeben: }
\begin{center}
\[
	v_{c,1} = 20 \frac{\text{m}}{\text{min}} \text{ , } v_{c,2} = 40 \frac{\text{m}}{\text{min}} \text{ , } v_{c,3} = 80 \frac{\text{m}}{\text{min}} \text{ , } v_{c,4} = 160 \frac{\text{m}}{\text{min}}
\]
\[
	D_{WZ,U} = 120 \text{mm; }
	D_{WZ,S} = 120 \text{mm }
\]
\[
	n_{an} = n_{\mathord{\mathrm{I}}}= 1200 \frac{1}{\text{min}}
\]
\[
	i_{1,2} =  i_{1,3} = -3
\]
 \begin{align*}
\text{Planetengetriebe: }
		&i_{0} = -3 \\	
		&n_{10} = 0\\
\end{align*}
\subsubsection{Berechnungen:}
\end{center}
Verwendete Formeln aus den Skripten KL II\ccite{bib:denkena:kl2} S. 34 und KL III\ccite{bib:poll:kl3} S. 2 \\
Mithilfe der Schnittgeschwindigkeiten werden die Abtriebsdrehzahlen der einzelnen Gänge bestimmt. Anschließend können dann die fehlenden Übersetzungen ermittelt werden.
\begin{itemize}
\item{Gang 1:}
\begin{align*}
	&n_{{\mathord{\mathrm{II}},1}} = \frac{n_{an}}{i_{1,2}} = 400 \frac{1}{\text{min}} \\
	&\text{Formel nach Willis: } n_{9}= \frac{n_8 - i_0 \x n_{10}}{1-i_0}\\
	&n_{9,1}= n_{\mathord{\mathrm{III}},1} = \frac{n_{8,1}}{1-i_0} = \frac{400 \frac{1}{\text{min}}}{1-(-3)} = 100 \frac{1}{\text{min}}\\
	&n_{\mathord{\mathrm{IV}},1} = n_{ab,1} =\frac{ v_{c,1}}{\pi \x D_{WZ}} = 53,05 \frac{1}{\text{min}} \\
	&\implies i_{6,7} = \frac{n_{{\mathord{\mathrm{III}},1}}}{n_{\mathord{\mathrm{IV}},1}} = -1,89 \\
\end{align*}
\item{Gang 2:}
\begin{align*}
	&n_{{\mathord{\mathrm{II}},2}} = \frac{n_{an}}{i_{1,2}} = 400 \frac{1}{\text{min}} \\
	&n_{\mathord{\mathrm{IV}},2} =n_{ab,2} = \frac{ v_{c,2}}{\pi \x D_{WZ}} =106,1 \frac{1}{\text{min}} \\
	&\implies i_{4,5} = \frac{ n_{\mathord{\mathrm{III}},2}}{n_{\mathord{\mathrm{IV}},2}} = 3,77 \\
\end{align*}
\item{Gang 3:}
\begin{align*}
		&n_{\mathord{\mathrm{III}},3} = \frac{n_{an}}{i_{1,3}} = 400 \frac{1}{\text{min}} \\
		&n_{\mathord{\mathrm{IV}},3} =\frac{ n_{\mathord{\mathrm{III}},3}}{i_{6,7}} =212,2 \frac{1}{\text{min}}\\	
		&\text{Überprüfung der Abtriebsdrehzahl: } n_{\mathord{\mathrm{IV}},3} =n_{ab,3} = \frac{ v_{c,3}}{\pi \x D_{WZ}} =212,2 \frac{1}{\text{min}} \\
\end{align*}
\item{Gang 4:}
\begin{align*}
	&n_{\mathord{\mathrm{III}},4} = n_{9,4}= \frac{n_{an}}{i_{1,3}} = 400 \frac{1}{\text{min}} \\
	&\text{Formel nach Willis: } n_{8}= i_0 \x n_{10} + n_9\x (1-i_0) \\
	&n_{8,4}= n_{\mathord{\mathrm{II}},4} = n_{9,4}\x (1-i_0) = 400 \frac{1}{\text{min}} \x (1- (-3)) = 1600 \frac{1}{\text{min}}\\
	&n_{\mathord{\mathrm{IV}},4} =\frac{ n_{\mathord{\mathrm{II}},4}}{i_{4,5}} =424,4 \frac{1}{\text{min}}\\
	&\text{Überprüfung der Abtriebsdrehzahl: } n_{\mathord{\mathrm{IV}},4} =n_{ab,4} = \frac{ v_{c,4}}{\pi \x D_{WZ}} =424,4 \frac{1}{\text{min}} \\
\end{align*}
\end{itemize}