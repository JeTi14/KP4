\newpage
\chapter{Lagerlebensdauerberechnung}
\section{Lagerkräfte im Gang 1}
\subsubsection{Berechnung der Zahnradkräfte}
Die Berechnung der Kräfte und die verwendeten Formeln entsprechen dem Kapitel 5.2. Im 1. Gang sind Z1/Z2 und Z6/Z7 im Eingriff.
\begin{itemize}
\item Gegebene Werte: 
	\begin{align*}
	&T_{\mathrm{II},1,S} = 18,4\text{ Nm} \\
	&T_{\mathrm{III},1,S} = 73,63\text{ Nm} \\	
	&d_2 = 217,1\text{ mm} \text{ , } d_6 = 149 \text{ mm , } d_8 = 112 \text{ mm}\\
	\end{align*}
\item Z2, Gang 1:
	\begin{align*} 
	&F_{t,2,1} = \frac{2\x T_{\mathrm{II},1,S}}{d_2} = \frac{2 \x 18,4 \text{ Nm}}{0,2171 \text{ m}} = 169,5 \text{ N}\\ 
	&F_{r,2,1} = \frac{169,5 \text{ N} \x \tan{(20^\circ)}}{\cos(20^\circ)} = 65,65\text{ N}\\ 
	&F_{a,2,1} =169,5 \x \tan(20^\circ) =61,7 \text{ N}
	\end{align*}
\item Z8, Gang 1:
	\begin{align*}
	&F_{t,8,1} = \frac{2\x T_{\mathrm{II},1,S}}{d_8 \x 4} = \frac{2 \x 18,4 \text{ Nm}}{0,112 \text{ m} \x 4} = 328,57\text{ N}\\ 
	&F_{r,8,1} = 328,57 \text{ N} \x \tan{(20^\circ)} = 115,59\text{ N}\\ 
	&F_{a,8,1} =0 \text{ N}\\
	\end{align*}
\item Z6, Gang 1:
	\begin{align*} 
	&F_{t,6,1} = \frac{2\x T_{\mathrm{III},1,S}}{d_6} = \frac{2 \x 73,63 \text{ Nm}}{0,149 \text{ m}} = 988,32 \text{ N}\\ 
	&F_{r,6,1} = \frac{988,32 \text{ N} \x \tan{(20^\circ)}}{\cos(20^\circ)} = 382,8\text{ N}\\ 
	&F_{a,6,1} =988,32 \x \tan(20^\circ) =359,72 \text{ N}
	\end{align*}
\end{itemize}
\subsubsection{Berechnung der Lagerkräfte}
\begin{itemize}
\item Momentensummen Hohlwelle:
\begin{align*}
	\sum M\textsubscript{y}\textsuperscript{(B)} &\overset{!}{=} 0 = -A_z \x l_{AB} - F_{t,6,1} \x l_{6B} \\
	&\implies A_z = -F_{t,6,1} \x \frac{l_{6B}}{l_{AB}} = -742,68 \text{ N} \\ \\
	\sum M\textsubscript{z}\textsuperscript{(B)} &\overset{!}{=} 0 = -A_y \x l_{AB} - F_{r,6,1} \x l_{6B} + F_{a,6,1} \x \frac{d_6}{2}\\
	&\implies A_y = \frac{-F_{r,6,1} \x l_{6B} + F_{a,6,1} \x \frac{d_6}{2}} {l_{AB}}= -130,9 \text{ N} 
\end{align*}
\item Gleichgewichte Welle II:
\begin{align*}
	\sum F_x &\overset{!}{=} 0 = F_{a,2,1} + D_x \implies D_x = -F_{a,2,1} = -61,7 \text{ N} \\
	\sum F_y &\overset{!}{=} 0 = C_y - F_{r,2,1}-A_y +D_y - F_{r,8} + F_{r,8}\\ 
	\sum F_z &\overset{!}{=} 0 = A_z - C_z - D_z - F_{t,2,1} + F_{t,8} - F_{t,8}\\ \\
	\sum M\textsubscript{y}\textsuperscript{(D)} &\overset{!}{=} 0 = A_z \x (l_3+l_4)- F_{t,2,1} \x (l_2+l_3+l_4) - C_z \x l_{ges} +l_4 \x F_{t,8}- l_4 \x F_{t,8} \\ 
	&\implies C_z = \frac{(l_3+l_4) \x A_z - F_{t,2,1} \x (l_2+l_3+l_4)}{l_{ges}} = -458,2 \text { N} \\ 
	& \implies D_z = A_z - C_z - F_{t,2,1}= -454 \text{ N}\\ \\
	\sum M\textsubscript{z}\textsuperscript{(D)} &\overset{!}{=} 0 = (l_3+l_4) \x A_y - \frac{d_2}{2} \x F_{a,2,1} + (l_2+l_3+l_4) \x F_{r,2,1}- l_{ges} \x C_y  \\ 
	&\implies C_y = \frac{(l_3+l_4) \x A_y - \frac{d_2}{2} \x F_{a,2,1} + (l_2+l_3+l_4) \x F_{r,2,1}}{l\textsubscript{ges}} = -34 \text{ N}\\ 
	& \implies D_y =   A_y - C_y + F_{r,2,1} = -31,25\text{ N}\\ 
\end{align*}
\begin{align*}
C_x &= \underline{0\text{ N}} & D_x= \underline{-61,7\text{ N}}\\
C_y &= \underline{-34\text{ N}} & D_y= \underline{-31,25\text{ N}}\\
C_z &= \underline{-458,2\text{ N}} & D_z= \underline{-454\text{ N}}\\
\end{align*}
\end{itemize}
\newpage
\section{Lagerkräfte im Gang 2}
\subsubsection{Berechnung der Zahnradkräfte}
Die Berechnung der Kräfte und die verwendeten Formeln entsprechen dem Kapitel 5.2. Im 2. Gang sind Z1/Z2 und Z4/Z5 im Eingriff.
\begin{itemize}
\item Gegebene Werte: 
	\begin{align*}
	&T_{\mathrm{II},2,S} = 48,96\text{ Nm} \\
	&d_2 = 217,1\text{ mm} \text{ , } d_4 = 89,4 \text{ mm } \\
	\end{align*}
\item Z2, Gang 2:
	\begin{align*} 
	&F_{t,2,2} = \frac{2\x T_{\mathrm{II},2,S}}{d_2} = \frac{2 \x 48,96 \text{ Nm}}{0,2171 \text{ m}} = 451 \text{ N}\\ 
	&F_{r,2,2} = \frac{451 \text{ N} \x \tan{(20^\circ)}}{\cos(20^\circ)} = 174,7\text{ N}\\ 
	&F_{a,2,2} =451 \x \tan(20^\circ) =164,15 \text{ N}
	\end{align*}
\item Z4, Gang 2:
	\begin{align*} 
	&F_{t,4,2} = \frac{2\x T_{\mathrm{II},2,S}}{d_4} = \frac{2 \x 48,96 \text{ Nm}}{0,0894 \text{ m}} = 1095,3 \text{ N}\\ 
	&F_{r,4,2} = \frac{1095,3 \text{ N} \x \tan{(20^\circ)}}{\cos(20^\circ)} = 424,2\text{ N}\\ 
	&F_{a,4,2} =1095,3 \x \tan(20^\circ) =398,7 \text{ N}
	\end{align*}
\end{itemize}
\subsubsection{Berechnung der Lagerkräfte}
\begin{itemize}
	\item Gleichgewichte Welle II:
	\begin{align*}
	\sum F_x &\overset{!}{=} 0 = F_{a,4,2} +F_{a,2,2} + D_x \implies D_x = -F_{a,4,2} -F_{a,2,2}= -562,85 \text{ N} \\
	\sum F_y &\overset{!}{=} 0 = C_y +F_{r,4,2}- F_{r,2,2}+D_y \\ 
	\sum F_z &\overset{!}{=} 0 = - C_z - D_z - F_{t,2,2} + F_{t,4,2} \\ \\
	\sum M\textsubscript{y}\textsuperscript{(D)} &\overset{!}{=} 0 = - F_{t,2,2} \x l_3 - C_z \x l_{ges} +(l_2 + l_3) \x F_{t,4,2} \\ 
	&\implies C_z = \frac{- F_{t,2,2} \x l_3 + (l_2+l_3) \x F_{t,4,2} }{l_{ges}} = 632,72 \text { N} \\ 
	& \implies D_z = - C_z - F_{t,2,2} + F_{t,4,2}= 11,58 \text{ N}\\ \\
	\sum M\textsubscript{z}\textsuperscript{(D)} &\overset{!}{=} 0 = - \frac{d_2}{2} \x F_{a,2,2} + \frac{d_4}{2} \x F_{a,4,2} + l_3 \x F_{r,2,2} - (l_2 + l_3)+ \x F_{r,4,2}- l_{ges} \x C_y  \\ 
	&\implies C_y = \frac{- \frac{d_2}{2} \x F_{a,2,2} + \frac{d_4}{2} \x F_{a,4,2}  + l_3 \x F_{r,2,2} - (l_2 + l_3) \x F_{r,4,2}}{l\textsubscript{ges}} = -245 \text{ N}\\ 
	& \implies D_y =  - C_y - F_{r,4,2} + F_{r,2,2} = -4,7\text{ N}\\ 
	\end{align*}
	\begin{align*}
	C_x &= \underline{0\text{ N}} & D_x= \underline{-562,85\text{ N}}\\
	C_y &= \underline{-245\text{ N}} & D_y= \underline{-4,7\text{ N}}\\
	C_z &= \underline{632,72\text{ N}} & D_z= \underline{11,58\text{ N}}\\
	\end{align*}
\end{itemize}
\section{Berechnung der äquivalenten dynamischen Lagerbelastung}
Die Formeln für die Berechnung stammen aus dem Skript KL III\ccite{bib:poll:kl3} Seite 73 bis 74.
Die Werte für die statische und dynamisch Tragzahl stammen aus dem Tabellenbuch Roloff/Matek \ccite{bib:roloffMatek:tabellenbuch} Seite 152.

\subsubsection{Gang 1, Festlager:} 2 Schrägkugellager in X-Anordnung, DIN 628-7205B\\
\begin{itemize}
	\item gegebene Werte:
	\begin{align*}
	&n_{an} &&= 1200 \frac{1}{\text{min}} \\
	&\text{statische Tragzahl Einzellager } C_{0,Einzel} &&= 9300 \text{ N}\\
	&\text{statische Tragzahl } C_0 &&= 2 \x C_{0,\text{Einzel}} = 18600 \text{ N} \\
	&\text{dynamische Tragzahl Einzellager } C_{Einzel} &&= 14600 \text{ N} \\
	&\text{dynamische Tragzahl } C &&= 1,62 \x C_{\text{Einzel}} = 23652 \text{ N}\\
	&\text{Lebensdauerexponent } p &&= 3 \text{ (für Wälzlager)} \\
	&F_{Ax} && = 64,63 \text{ N}\\
	&F_{Ay} && = 243,89 \text{ N}\\
	&F_{Az} && = 2212,11 \text{ N}\\
	\end{align*} 
	\item Berechnung der äquivalenten dynamischen Belastung
	\begin{align*}
	&\text{dynamische radiale Lagerkraft } F_r&& = \sqrt{F_{Ax}^2 + F_{Ay}^2 } = 252,3 \text{ N} \\
	&\text{dynamische axiale Lagerkraft } F_a&& = F_{Az} = 2212,11 \text{ N}\\
	&\text{Belastungsfaktor } e &&= 1,14 \text{ (für Schrägkugellager)} \\
	\end{align*} 
	\[\frac{F_a}{F_r} = 8,77 \implies \frac{F_a}{F_r} > e\]
	Deshalb folgt aus Tabelle 2.9 im Skript KL III\ccite{bib:poll:kl3} : X= 0,57 \text{, } Y= 0,93 \\
	Die äquivalente dynamische Belastung ergibt sich zu: 
	\[
	P= X \x F_r + Y \x F_a = 2201,1 \text{ N}
	\]
\end{itemize}

\subsubsection{Gang 1, Loslager:} Rillenkugellager, DIN 625-6007\\
\begin{itemize}
	\item gegebene Werte:
	\begin{align*}
	&n_{an}&&= 1200 \frac{1}{\text{min}} \\
	&\text{statische Tragzahl } C_0 &&= 10200 \text{ N}\\
	&\text{dynamische Tragzahl } C &&= 16000 \text{ N} \\
	&\text{Lebensdauerexponent } p&&= 3  \\
	&F_{Bx} && = 79,64 \text{ N}\\
	&F_{By} && = 247,83 \text{ N}\\
	\end{align*} 
	\item Berechnung der äquivalenten dynamischen Belastung
	\begin{align*}
	&\text{dynamische radiale Lagerkraft } F_r&& = \sqrt{F_{Bx}^2 + F_{By}^2 } = 260,31 \text{ N} \\
	&\text{dynamische axiale Lagerkraft } F_a&& = F_{Bz} = 0\text{ N}\\
	\end{align*} 
	Da es sich um eine reine Radialbelastung handelt, ergeben sich der Radial- und Axialfaktor zu: $X= 1$ und $Y=0$\\
	Daraus ergibt sich die äquivalente dynamische Lagerbelastung zu:  
	\[
	P= X \x F_r =260,31 \text{ N}
	\]
\end{itemize}
\newpage

\subsubsection{Gang 2, Festlager:} 2 Schrägkugellager in X-Anordnung, DIN 628-7205B\\
\begin{itemize}
	\item gegebene Werte:
	\begin{align*}
	&n_{an}&&= 1200 \frac{1}{\text{min}} \\
	&\text{statische Tragzahl Einzellager } C_{0,Einzel} &&= 9300 \text{ N}\\
	&\text{statische Tragzahl } C_0 &&= 2 \x C_{0,\text{Einzel}} = 18600 \text{ N} \\
	&\text{dynamische Tragzahl Einzellager } C_{Einzel} &&= 14600 \text{ N} \\
	&\text{dynamische Tragzahl } C &&= 1,62 \x C_{\text{Einzel}} = 23652 \text{ N}\\
	&\text{Lebensdauerexponent } p&&= 3  \\
	&F_{Ax} && = 145,44 \text{ N}\\
	&F_{Ay} && = 548,63 \text{ N}\\
	&F_{Az} && = 2078,59 \text{ N}\\
	\end{align*} 
	\item Berechnung der äquivalenten dynamischen Belastung
	\begin{align*}
	&\text{dynamische radiale Lagerkraft } F_r&& = \sqrt{F_{Ax}^2 + F_{Ay}^2 } = 567,58 \text{ N} \\
	&\text{dynamische axiale Lagerkraft } F_a&& = F_{Az} = 2078,59 \text{ N}\\
	&\text{Belastungsfaktor } e &&= 1,14 \text{ (für Schrägkugellager)} \\
	\end{align*} 
	\[\frac{F_a}{F_r}= 3,66 \implies \frac{F_a}{F_r} > e\]
	Deshalb folgt aus Tabelle 2.9 im Skript KL III\ccite{bib:poll:kl3} : X= 0,57 \text{, } Y= 0,93 \\
	Die äquivalente dynamische Belastung ergibt sich zu: 
	\[
		P= X \x F_r + Y \x F_a = 2256,6 \text{ N}
	\]
\end{itemize}
\newpage

\subsubsection{Gang 2, Loslager:} Rillenkugellager, DIN 625-6007\\
\begin{itemize}
	\item gegebene Werte:
	\begin{align*}
	&n_{an}&&= 1200 \frac{1}{\text{min}} \\
	&\text{statische Tragzahl } C_0 &&= 10200 \text{ N}\\
	&\text{dynamische Tragzahl } C &&= 16000 \text{ N} \\
	&\text{Lebensdauerexponent } p&&= 3  \\
	&F_{Bx} && = 180,11 \text{ N}\\
	&F_{By} && = 560,8 \text{ N}\\
	\end{align*} 
	\item Berechnung der äquivalenten dynamischen Belastung
	\begin{align*}
	&\text{dynamische radiale Lagerkraft } F_r&& = \sqrt{F_{Bx}^2 + F_{By}^2 } = 589 \text{ N} \\
	&\text{dynamische axiale Lagerkraft } F_a&& = F_{Bz} = 0\text{ N}\\
	\end{align*} 
	Da es sich um eine reine Radialbelastung handelt, ergeben sich der Radial- und Axialfaktor zu: $X= 1$ und $Y=0$\\
	Daraus ergibt sich die äquivalente dynamische Lagerbelastung zu:  
	\[
	P= X \x F_r = 589 \text{ N}
	\]
\end{itemize}
\newpage

\section{Bestimmung der Lastkollektive und der Lebensdauer}
Die Formeln für die Berechnung stammen aus dem Skript KL III\ccite{bib:poll:kl3} Seite 76
\subsubsection{Lastkollektiv Festlager}
\begin{align*}
	n_m &= 1200 \frac{1}{min} \text{, da die Drehzahl in beiden Gängen gleich ist} \\
	P_{\text{äq}} &= \sqrt[p]{P_1 ^{p} \x \frac{n_1}{n_m} \x \frac{q_1}{100\% } + P_2 ^{p} \x \frac{n_2}{n_m} \x \frac{q_2}{100 \%}} \text{ mit $q_1 = 70$ und $q_2= 30$} \\
	&= \sqrt[3]{(2201,1 \text{ N}) ^{3} \x \frac{1200}{1200} \x \frac{70 \%}{100 \%} +(2256,6 \text{ N} )^{3} \x \frac{1200}{1200} \x \frac{30 \%}{100\% }} = 2218,04 \text{ N}\\
\end{align*}
\subsubsection{Lastkollektiv Loslager}
\begin{align*}
	n_m &= 1200 \frac{1}{min} \text{, da die Drehzahl in beiden Gängen gleich ist} \\
	P_{\text{äq}} &= \sqrt[p]{P_1 ^{p} \x \frac{n_1}{n_m} \x \frac{q_1}{100\% } + P_2 ^{p} \x \frac{n_2}{n_m} \x \frac{q_2}{100 \%}} \text{ mit $q_1 = 70$ und $q_2= 30$} \\
	&= \sqrt[3]{(260,31 \text{ N}) ^{3} \x \frac{1200}{1200} \x \frac{70 \%}{100 \%} +(589 \text{ N}) ^{3} \x \frac{1200}{1200} \x \frac{30 \%}{100\% }} = 419,17 \text{ N}\\
\end{align*}

\subsubsection{Lebensdauer Festlager}
$L_{10h}= \left( \frac{C}{P_{\text{äq}}} \right) ^p \x \frac{10^6}{n \x 60} = \left( \frac{23652 \text{ N}}{2218,04 \text{ N}} \right) ^3 \x \frac{10^6}{1200 \frac{1}{\text{min}} \x 60} = 16840,81 \text{ h} = 1,92 \text{ a}$

\subsubsection{Lebensdauer Festlager}
$L_{10h}= \left( \frac{C}{P_{\text{äq}}} \right) ^p \x \frac{10^6}{n \x 60} = \left( \frac{16000 \text{ N}}{419,17 \text{ N}} \right) ^3 \x \frac{10^6}{1200 \frac{1}{\text{min}} \x 60} = 772425,98 \text{ h} = 88,18 \text{ a}$