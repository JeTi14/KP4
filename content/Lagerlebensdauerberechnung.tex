\newpage
\chapter{Lagerlebensdauerberechnung}
\section{Lagerkräfte im Gang 1}
\subsubsection{Berechnung der Zahnradkräfte}
Die Berechnung der Kräfte und die verwendeten Formeln entsprechen dem Kapitel 3.2. Im 1. Gang sind Z1/Z2 und Z6/Z7 im Eingriff.
\begin{itemize}
\item Gegebene Werte: 
	\begin{align*}
	&T_{\mathrm{II},1,S} = 18,4\text{ Nm} \\
	&T_{\mathrm{III},1,S} = 73,63\text{ Nm} \\	
	&d_2 = 217,1\text{ mm} \text{ , } d_6 = 149 \text{ mm , } d_8 = 112 \text{ mm}
	\end{align*}
\item Z2, Gang 1:
	\begin{align*} 
	&F_{t,2,1} = \frac{2\x T_{\mathrm{II},1,S}}{d_2} = \frac{2 \x 18,4 \text{ Nm}}{0,2171 \text{ m}} = 169,5 \text{ N}\\ 
	&F_{r,2,1} = \frac{169,5 \text{ N} \x \tan{(20^\circ)}}{\cos(20^\circ)} = 65,65\text{ N}\\ 
	&F_{a,2,1} =169,5 \x \tan(20^\circ) =61,7 \text{ N}
	\end{align*}
\item Z8, Gang 1:
	\begin{align*}
	&F_{t,8,1} = \frac{2\x T_{\mathrm{II},1,S}}{d_8 \x 4} = \frac{2 \x 18,4 \text{ Nm}}{0,112 \text{ m} \x 4} = 328,57\text{ N}\\ 
	&F_{r,8,1} = 328,57 \text{ N} \x \tan{(20^\circ)} = 115,59\text{ N}\\ 
	&F_{a,8,1} =0 \text{ N}
	\end{align*}
\item Z6, Gang 1:
	\begin{align*} 
	&F_{t,6,1} = \frac{2\x T_{\mathrm{III},1,S}}{d_6} = \frac{2 \x 73,63 \text{ Nm}}{0,149 \text{ m}} = 988,32 \text{ N}\\ 
	&F_{r,6,1} = \frac{988,32 \text{ N} \x \tan{(20^\circ)}}{\cos(20^\circ)} = 382,8\text{ N}\\ 
	&F_{a,6,1} =988,32 \x \tan(20^\circ) =359,72 \text{ N}
	\end{align*}
\end{itemize}
\subsubsection{Berechnung der Lagerkräfte}
\begin{itemize}
\item Momentensummen Hohlwelle:
\begin{align*}
	\sum M\textsubscript{y}\textsuperscript{(B)} &\overset{!}{=} 0 = -A_z \x l_{AB} - F_{t,6,1} \x l_{6B} \\
	&\implies A_z = -F_{t,6,1} \x \frac{l_{6B}}{l_{AB}} = -742,68 \text{ N} \\ \\
	\sum M\textsubscript{z}\textsuperscript{(B)} &\overset{!}{=} 0 = -A_y \x l_{AB} - F_{r,6,1} \x l_{6B} + F_{a,6,1} \x \frac{d_6}{2}\\
	&\implies A_y = \frac{-F_{r,6,1} \x l_{6B} + F_{a,6,1} \x \frac{d_6}{2}} {l_{AB}}= -130,9 \text{ N} 
\end{align*}
\item Gleichgewichte Welle II:
\begin{align*}
	\sum F_x &\overset{!}{=} 0 = F_{a,2,1} + D_x \implies D_x = -F_{a,2,1} = -61,7 \text{ N} \\
	\sum F_y &\overset{!}{=} 0 = C_y - F_{r,2,1}-A_y +D_y - F_{r,8} + F_{r,8}\\ 
	\sum F_z &\overset{!}{=} 0 = A_z - C_z - D_z - F_{t,2,1} + F_{t,8} - F_{t,8}\\ \\
	\sum M\textsubscript{y}\textsuperscript{(D)} &\overset{!}{=} 0 = A_z \x (l_3+l_4)- F_{t,2,1} \x (l_2+l_3+l_4) - C_z \x l_{ges} +l_4 \x F_{t,8}- l_4 \x F_{t,8} \\ 
	&\implies C_z = \frac{(l_3+l_4) \x A_z - F_{t,2,1} \x (l_2+l_3+l_4)}{l_{ges}} = -458,2 \text { N} \\ 
	& \implies D_z = A_z - C_z - F_{t,2,1}= -454 \text{ N}\\ \\
	\sum M\textsubscript{z}\textsuperscript{(D)} &\overset{!}{=} 0 = (l_3+l_4) \x A_y - \frac{d_2}{2} \x F_{a,2,1} + (l_2+l_3+l_4) \x F_{r,2,1}- l_{ges} \x C_y  \\ 
	&\implies C_y = \frac{(l_3+l_4) \x A_y - \frac{d_2}{2} \x F_{a,2,1} + (l_2+l_3+l_4) \x F_{r,2,1}}{l\textsubscript{ges}} = -34 \text{ N}\\ 
	& \implies D_y =   A_y - C_y + F_{r,2,1} = -31,25\text{ N}
\end{align*}
\begin{align*}
	C_{x,1} &= \underline{0\text{ N}} & D_{x,1}= \underline{-61,7\text{ N}}\\
	C_{y,1} &= \underline{-34\text{ N}} & D_{y,1}= \underline{-31,25\text{ N}}\\
	C_{z,1} &= \underline{-458,2\text{ N}} & D_{z,1}= \underline{-454\text{ N}}
\end{align*}
\end{itemize}
\section{Lagerkräfte im Gang 2}
\subsubsection{Berechnung der Zahnradkräfte}
Die Berechnung der Kräfte und die verwendeten Formeln entsprechen dem Kapitel 3.2. Im 2. Gang sind Z1/Z2 und Z4/Z5 im Eingriff.
\begin{itemize}
\item Gegebene Werte: 
	\begin{align*}
	&T_{\mathrm{II},2,S} = 48,96\text{ Nm} \\
	&d_2 = 217,1\text{ mm} \text{ , } d_4 = 89,4 \text{ mm } 
	\end{align*}
\item Z2, Gang 2:
	\begin{align*} 
	&F_{t,2,2} = \frac{2\x T_{\mathrm{II},2,S}}{d_2} = \frac{2 \x 48,96 \text{ Nm}}{0,2171 \text{ m}} = 451 \text{ N}\\ 
	&F_{r,2,2} = \frac{451 \text{ N} \x \tan{(20^\circ)}}{\cos(20^\circ)} = 174,7\text{ N}\\ 
	&F_{a,2,2} =451 \x \tan(20^\circ) =164,15 \text{ N}
	\end{align*}
\item Z4, Gang 2:
	\begin{align*} 
	&F_{t,4,2} = \frac{2\x T_{\mathrm{II},2,S}}{d_4} = \frac{2 \x 48,96 \text{ Nm}}{0,0894 \text{ m}} = 1095,3 \text{ N}\\ 
	&F_{r,4,2} = \frac{1095,3 \text{ N} \x \tan{(20^\circ)}}{\cos(20^\circ)} = 424,2\text{ N}\\ 
	&F_{a,4,2} =1095,3 \x \tan(20^\circ) =398,7 \text{ N}
	\end{align*}
\end{itemize}
\subsubsection{Berechnung der Lagerkräfte}
\begin{itemize}
	\item Gleichgewichte Welle II:
	\begin{align*}
	\sum F_x &\overset{!}{=} 0 = F_{a,4,2} +F_{a,2,2} + D_x \implies D_x = -F_{a,4,2} -F_{a,2,2}= -562,85 \text{ N} \\
	\sum F_y &\overset{!}{=} 0 = C_y +F_{r,4,2}- F_{r,2,2}+D_y \\ 
	\sum F_z &\overset{!}{=} 0 = - C_z - D_z - F_{t,2,2} + F_{t,4,2} \\ \\
	\sum M\textsubscript{y}\textsuperscript{(D)} &\overset{!}{=} 0 = - F_{t,2,2} \x l_3 - C_z \x l_{ges} +(l_2 + l_3) \x F_{t,4,2} \\ 
	&\implies C_z = \frac{- F_{t,2,2} \x l_3 + (l_2+l_3) \x F_{t,4,2} }{l_{ges}} = 632,72 \text { N} \\ 
	& \implies D_z = - C_z - F_{t,2,2} + F_{t,4,2}= 11,58 \text{ N}\\ \\
	\sum M\textsubscript{z}\textsuperscript{(D)} &\overset{!}{=} 0 = - \frac{d_2}{2} \x F_{a,2,2} + \frac{d_4}{2} \x F_{a,4,2} + l_3 \x F_{r,2,2} - (l_2 + l_3)+ \x F_{r,4,2}- l_{ges} \x C_y  \\ 
	&\implies C_y = \frac{- \frac{d_2}{2} \x F_{a,2,2} + \frac{d_4}{2} \x F_{a,4,2}  + l_3 \x F_{r,2,2} - (l_2 + l_3) \x F_{r,4,2}}{l\textsubscript{ges}} = -245 \text{ N}\\ 
	& \implies D_y =  - C_y - F_{r,4,2} + F_{r,2,2} = -4,7\text{ N}
	\end{align*}
	\begin{align*}
	C_{x,2} &= \underline{0\text{ N}} & D_{x,2}= \underline{-562,85\text{ N}}\\
	C_{y,2} &= \underline{-245\text{ N}} & D_{y,2}= \underline{-4,7\text{ N}}\\
	C_{z,2} &= \underline{632,72\text{ N}} & D_{z,2}= \underline{11,58\text{ N}}
	\end{align*}
\end{itemize}
\section{Lagerkräfte im Gang 4}
Die Lagerkräfte der Welle 2 im 4. Gang wurden im Kaptel 3.2 zur Berechnung der Schnittkräfte bestimmt.
\begin{align*}
	C_{x,4} &= \underline{0\text{ N}} & D_{x,4}= \underline{-581,94\text{ N}}\\
	C_{y,4} &= \underline{-258\text{ N}} & D_{y,4}= \underline{78,9\text{ N}}\\
	C_{z,4} &= \underline{-2616,8\text{ N}} & D_{z,4}= \underline{-1689,53\text{ N}}
\end{align*}
\section{Berechnung der äquivalenten dynamischen Lagerbelastung}
Die Formeln für die Berechnung stammen aus dem Skript KL III\ccite{bib:poll:kl3} Seite 73 bis 74.
Die Werte für die statische und dynamisch Tragzahl stammen aus dem Tabellenbuch Roloff/Matek \ccite{bib:roloffMatek:tabellenbuch} Seite 205.
\subsection{Gang 1}
\subsubsection{Gang 1, Festlager:} Rillenkugellager , DIN 625-6006\\
\begin{itemize}
	\item gegebene Werte:
	\begin{align*}
	&n_{{\mathord{\mathrm{II}},1}} &&=  400 \frac{1}{\text{min}} \\
	&\text{statische Tragzahl } C_{0} &&= 8000 \text{ N}\\
	&\text{dynamische Tragzahl } C &&= 12700 \text{ N} \\
	&\text{Lebensdauerexponent } p &&= 3 \text{ (für Wälzlager)} \\
	&F_{Dx} && = -61,7 \text{ N}\\
	&F_{Dy} && = 31,25 \text{ N}\\
	&F_{Dz} && = -454 \text{ N}
	\end{align*} 
	\item Berechnung der äquivalenten dynamischen Belastung
	\begin{align*}
	&\text{dynamische radiale Lagerkraft } F_r&& = \sqrt{F_{Dy}^2 + F_{Dz}^2 } = 455,1 \text{ N} \\
	&\text{dynamische axiale Lagerkraft } F_a&& = |F_{Dx}| = 61,7 \text{ N}\\
	&\text{Belastungsfaktor } e &&= 0,22 \text{ da } \frac{F_a}{C_0} < 0,025
	\end{align*} 
	\[\frac{F_a}{F_r} = 0,14 \implies \frac{F_a}{F_r} < e\]
	Deshalb folgt aus Tabelle 2.9 im Skript KL III\ccite{bib:poll:kl3} : X= 1 \text{, } Y= 0 \\
	Die äquivalente dynamische Belastung ergibt sich zu: 
	\[
	P= X \x F_r = 455,1 \text{ N}
	\]
\end{itemize}

\subsubsection{Gang 1, Loslager:} Rillenkugellager, DIN 625-6006\\
\begin{itemize}
	\item gegebene Werte:
	\begin{align*}
	&n_{{\mathord{\mathrm{II}},1}} &&=  400 \frac{1}{\text{min}} \\
	&\text{statische Tragzahl } C_{0} &&= 8000 \text{ N}\\
	&\text{dynamische Tragzahl } C &&= 12700 \text{ N} \\
	&\text{Lebensdauerexponent } p &&= 3 \text{ (für Wälzlager)} \\
	&F_{Cy} && = -34 \text{ N}\\
	&F_{Cz} && = -458,2 \text{ N}
	\end{align*} 
	\item Berechnung der äquivalenten dynamischen Belastung
	\begin{align*}
	&\text{dynamische radiale Lagerkraft } F_r&& = \sqrt{F_{Cy}^2 + F_{Cz}^2 } = 459,6 \text{ N} \\
	&\text{dynamische axiale Lagerkraft } F_a&& = F_{Cx} = 0\text{ N}
	\end{align*} 
	Da es sich um eine reine Radialbelastung handelt, ergeben sich der Radial- und Axialfaktor zu: $X= 1$ und $Y=0$\\
	Daraus ergibt sich die äquivalente dynamische Lagerbelastung zu:  
	\[
	P= X \x F_r =459,6 \text{ N}
	\]
\end{itemize}
\newpage

\subsection{Gang 2}
\subsubsection{Gang 2, Festlager:} Rillenkugellager , DIN 625-6006\\
\begin{itemize}
	\item gegebene Werte:
	\begin{align*}
	&n_{{\mathord{\mathrm{II}},2}} &&=  400 \frac{1}{\text{min}} \\
	&\text{statische Tragzahl } C_{0} &&= 8000 \text{ N}\\
	&\text{dynamische Tragzahl } C &&= 12700 \text{ N} \\
	&\text{Lebensdauerexponent } p &&= 3 \text{ (für Wälzlager)} \\
	&F_{Dx} && = -562,58 \text{ N}\\
	&F_{Dy} && = -4,7 \text{ N}\\
	&F_{Dz} && = 11,58 \text{ N}
	\end{align*} 
	\item Berechnung der äquivalenten dynamischen Belastung
	\begin{align*}
	&\text{dynamische radiale Lagerkraft } F_r&& = \sqrt{F_{Dy}^2 + F_{Dz}^2 } = 12,5 \text{ N} \\
	&\text{dynamische axiale Lagerkraft } F_a&& = |F_{Dx}| = 562,58 \text{ N}\\
	&\text{Belastungsfaktor } e &&= 0,27 \text{ da } \frac{F_a}{C_0} = 0,07
	\end{align*}
	\[\frac{F_a}{F_r} = 45 \implies \frac{F_a}{F_r} > e\]
	Deshalb folgt aus Tabelle 2.9 im Skript KL III\ccite{bib:poll:kl3} : X= 0,56 \text{, } Y= 1,6 \\
	Die äquivalente dynamische Belastung ergibt sich zu: 
	\[
	P= X \x F_r + Y \x F_a = 907,1 \text{ N}
	\]
\end{itemize}
\newpage
\subsubsection{Gang 2, Loslager:} Rillenkugellager, DIN 625-6006\\
\begin{itemize}
	\item gegebene Werte:
	\begin{align*}
	&n_{{\mathord{\mathrm{II}},2}} &&=  400 \frac{1}{\text{min}} \\
	&\text{statische Tragzahl } C_{0} &&= 8000 \text{ N}\\
	&\text{dynamische Tragzahl } C &&= 12700 \text{ N} \\
	&\text{Lebensdauerexponent } p &&= 3 \text{ (für Wälzlager)} \\
	&F_{Cy} && = -245 \text{ N}\\
	&F_{Cz} && = 632,72 \text{ N}
	\end{align*} 
	\item Berechnung der äquivalenten dynamischen Belastung
	\begin{align*}
	&\text{dynamische radiale Lagerkraft } F_r&& = \sqrt{F_{Cy}^2 + F_{Cz}^2 } =678,5 \text{ N} \\
	&\text{dynamische axiale Lagerkraft } F_a&& = F_{Cx} = 0\text{ N}
	\end{align*} 
	Da es sich um eine reine Radialbelastung handelt, ergeben sich der Radial- und Axialfaktor zu: $X= 1$ und $Y=0$\\
	Daraus ergibt sich die äquivalente dynamische Lagerbelastung zu:  
	\[
	P= X \x F_r =678,5 \text{ N}
	\]
\end{itemize}
\newpage

\subsection{Gang 4}
\subsubsection{Gang 4, Festlager:} Rillenkugellager , DIN 625-6006\\
\begin{itemize}
	\item gegebene Werte:
	\begin{align*}
	&n_{{\mathord{\mathrm{II}},4}} &&=  1600 \frac{1}{\text{min}} \\
	&\text{statische Tragzahl } C_{0} &&= 8000 \text{ N}\\
	&\text{dynamische Tragzahl } C &&= 12700 \text{ N} \\
	&\text{Lebensdauerexponent } p &&= 3 \text{ (für Wälzlager)} \\
	&F_{Dx} && = -581,94 \text{ N}\\
	&F_{Dy} && = 78,9 \text{ N}\\
	&F_{Dz} && = -1689,53 \text{ N}
	\end{align*} 
	\item Berechnung der äquivalenten dynamischen Belastung
	\begin{align*}
	&\text{dynamische radiale Lagerkraft } F_r&& = \sqrt{F_{Dy}^2 + F_{Dz}^2 } =1691,4\text{ N} \\
	&\text{dynamische axiale Lagerkraft } F_a&& = |F_{Dx}| = 581,94 \text{ N}\\
	&\text{Belastungsfaktor } e &&= 0,27 \text{ da } \frac{F_a}{C_0} = 0,07
	\end{align*} 
	\[\frac{F_a}{F_r} = 0,34 \implies \frac{F_a}{F_r} > e\]
	Deshalb folgt aus Tabelle 2.9 im Skript KL III\ccite{bib:poll:kl3} : X= 0,56 \text{, } Y= 1,6 \\
	Die äquivalente dynamische Belastung ergibt sich zu: 
	\[
	P= X \x F_r + Y \x F_a= 1878,3 \text{ N}
	\]
\end{itemize}
\newpage
\subsubsection{Gang 4, Loslager:} Rillenkugellager, DIN 625-6006\\
\begin{itemize}
	\item gegebene Werte:
	\begin{align*}
	&n_{{\mathord{\mathrm{II}},2}} &&=  1600 \frac{1}{\text{min}} \\
	&\text{statische Tragzahl } C_{0} &&= 8000 \text{ N}\\
	&\text{dynamische Tragzahl } C &&= 12700 \text{ N} \\
	&\text{Lebensdauerexponent } p &&= 3 \text{ (für Wälzlager)} \\
	&F_{Cy} && = -258 \text{ N}\\
	&F_{Cz} && = -2616,8 \text{ N}
	\end{align*} 
	\item Berechnung der äquivalenten dynamischen Belastung
	\begin{align*}
	&\text{dynamische radiale Lagerkraft } F_r&& = \sqrt{F_{Cy}^2 + F_{Cz}^2 } =2629,5 \text{ N} \\
	&\text{dynamische axiale Lagerkraft } F_a&& = F_{Cx} = 0\text{ N}
	\end{align*} 
	Da es sich um eine reine Radialbelastung handelt, ergeben sich der Radial- und Axialfaktor zu: $X= 1$ und $Y=0$\\
	Daraus ergibt sich die äquivalente dynamische Lagerbelastung zu:  
	\[
	P= X \x F_r =2629,5\text{ N}
	\]
\end{itemize}
\newpage

\section{Bestimmung der Lastkollektive und der Lebensdauer}
Die Formeln für die Berechnung stammen aus dem Skript KL III\ccite{bib:poll:kl3} Seite 76
\subsubsection{Lastkollektiv Festlager}
\begin{align*}
	n_m &= n_{\mathord{\mathrm{II}},1} \x \frac{40 \%}{100 \%} + n_{\mathord{\mathrm{II}},2} \x \frac{25 \%}{100 \%} + n_{\mathord{\mathrm{II}},3} \x \frac{20 \%}{100 \%} + n_{\mathord{\mathrm{II}},4} \x \frac{15 \%}{100 \%} \\
	n_m &= 400 \frac{1}{\text{min}} \x 0,4 + 400 \frac{1}{\text{min}} \x 0,25 + 0 \frac{1}{\text{min}} + 1600 \frac{1}{\text{min}} \x 0,15 = 500 \frac{1}{\text{min}} \\
	P_{\text{äq}} &= \sqrt[p]{P_1 ^{p} \x \frac{n_1}{n_m} \x \frac{q_1}{100\% } + P_2 ^{p} \x \frac{n_2}{n_m} \x \frac{q_2}{100 \%}} \text{ mit $q_1 = 70$ und $q_2= 30$} \\
	&= \sqrt[3]{(455,1 \text{ N}) ^{3} \x \frac{400}{500} \x \frac{40 \%}{100 \%} +(907,1 \text{ N} )^{3} \x \frac{400}{500} \x \frac{25 \%}{100\% } +(1878,3 \text{ N} )^{3} \x \frac{1600}{500} \x \frac{15 \%}{100\% }} \\
	&= 1497,8 \text{ N}
\end{align*}
\subsubsection{Lastkollektiv Loslager}
\begin{align*}
	n_m &= n_{\mathord{\mathrm{II}},1} \x \frac{40 \%}{100 \%} + n_{\mathord{\mathrm{II}},2} \x \frac{25 \%}{100 \%} + n_{\mathord{\mathrm{II}},3} \x \frac{20 \%}{100 \%} + n_{\mathord{\mathrm{II}},4} \x \frac{15 \%}{100 \%} \\
	n_m &= 400 \frac{1}{\text{min}} \x 0,4 + 400 \frac{1}{\text{min}} \x 0,25 + 0 \frac{1}{\text{min}} + 1600 \frac{1}{\text{min}} \x 0,15 = 500 \frac{1}{\text{min}} \\
	P_{\text{äq}} &= \sqrt[p]{P_1 ^{p} \x \frac{n_1}{n_m} \x \frac{q_1}{100\% } + P_2 ^{p} \x \frac{n_2}{n_m} \x \frac{q_2}{100 \%}} \text{ mit $q_1 = 70$ und $q_2= 30$} \\
	&= \sqrt[3]{(459,6 \text{ N}) ^{3} \x \frac{400}{500} \x \frac{40 \%}{100 \%} +(678,5 \text{ N} )^{3} \x \frac{400}{500} \x \frac{25 \%}{100\% } +(2629,5 \text{ N} )^{3} \x \frac{1600}{500} \x \frac{15 \%}{100\% }} \\
	&= 2066,2 \text{ N}
\end{align*}

\subsubsection{Lebensdauer Festlager}
$L_{10h}= \left( \frac{C}{P_{\text{äq}}} \right) ^p \x \frac{10^6}{n_m \x 60} = \left( \frac{12700 \text{ N}}{1497,8 \text{ N}} \right) ^3 \x \frac{10^6}{500 \frac{1}{\text{min}} \x 60} = 20320,2 \text{ h} = 2,3 \text{ a}$

\subsubsection{Lebensdauer Festlager}
$L_{10h}= \left( \frac{C}{P_{\text{äq}}} \right) ^p \x \frac{10^6}{n_m \x 60} = \left( \frac{12700 \text{ N}}{2066,2 \text{ N}} \right) ^3 \x \frac{10^6}{500 \frac{1}{\text{min}} \x 60} = 7740,6 \text{ h} = 0,88 \text{ a}$