\section{Endgültige Abmaße}
Bei der Konstruktion haben sich aufgrund der vorgegebenen Achsabstände und anderen konstruktiven Gründen einige Änderungen in den endgültigen Abmaßen der Zahnräder verändert. Deshalb werden im Folgenden alle endgültigen Werte dargestellt.
\subsubsection{Stirnräder:}
\begin{align*}
	d_1 &= 72,36\text{ mm} & b_1= 32\text{ mm}\\
	d_2 &= 217,1\text{ mm} & b_2= 30\text{ mm}\\
	d_3 &= 217,1\text{ mm} & b_3= 30\text{ mm}\\
	d_4 &= 89,4\text{ mm} & b_4= 32\text{ mm}\\
	d_5 &= 340,5\text{ mm} & b_5= 30\text{ mm}\\
	d_6 &= 149\text{ mm} & b_6= 32\text{ mm}\\
	d_7 &= 280,9\text{ mm} & b_7= 30\text{ mm}\\
	d_8 &= 112\text{ mm} & b_8= 35\text{ mm}\\
	d_9 &= 112\text{ mm} & b_9= 28\text{ mm}\\
	d_{10} &= 336\text{ mm} & b_{10}= 30\text{ mm}
\end{align*}
\subsubsection{Kegelräder:}
\begin{align*}
	&d_{e,11} = d_{e,12} = 120\text{ mm} \\
	&b_{11} = b_{12} = 28 \text{ mm} \\
	&R_e = 84,85 \text{ mm}
\end{align*}
\subsubsection{Überprüfung der Gesamtübersetzung:}
\begin{itemize}
\item geforderte Gesamtübersetzung:
\begin{align*}
	&i_{ges,1} = \frac{n_{an}}{n_{\mathrm{IV},1}} = \frac{1200 \frac{1}{\text{min}}}{53,05 \frac{1}{\text{min}}} = 22,62 \\
	&i_{ges,2} = \frac{n_{an}}{n_{\mathrm{IV},2}} = \frac{1200 \frac{1}{\text{min}}}{106,1 \frac{1}{\text{min}}} = 11,31 \\
	&i_{ges,3} = \frac{n_{an}}{n_{\mathrm{IV},3}} = \frac{1200 \frac{1}{\text{min}}}{212,2 \frac{1}{\text{min}}} = 5,66 \\
	&i_{ges,4} = \frac{n_{an}}{n_{\mathrm{IV},4}} = \frac{1200 \frac{1}{\text{min}}}{424,4 \frac{1}{\text{min}}} = 2,83
\end{align*}

\item tatsächliche Gesamtübersetzung:
	\begin{align*}
	&i_{\tilde{ges,1}} = i_{1,2} \cdot i_{8,9} \cdot i_{6,7} = \frac{z_2}{z_1} \x \cdot i_{8,9} \x \frac{z_{7}}{z_{6}} = \frac{57}{19} \x 4 \x \frac{66}{35}= 22,63 \\
	&i_{\tilde{ges,2}} = i_{1,2} \cdot i_{4,5} = \frac{z_2}{z_1} \x \frac{z_{5}}{z_{4}} = \frac{57}{19} \x \frac{80}{21} = 11,4 \\
	&i_{\tilde{ges,3}} = i_{1,3} \cdot i_{6,7} = \frac{z_3}{z_1} \x \frac{z_{7}}{z_{6}} = \frac{57}{19} \x \frac{66}{35} = 5,66\\
	&i_{\tilde{ges,4}} = i_{1,3} \cdot i_{9,8} \cdot i_{4,5} = \frac{z_3}{z_1} \cdot i_{9,8} \x \frac{z_{5}}{z_{4}} = \frac{57}{19} \x  0,25 \x \frac{80}{21} = 2,857
	\end{align*}
\end{itemize}
maximale prozentuale Abweichung im Gang 4: $\frac{2,857}{2,83} = 1,00954\\
\implies 0,95 \% $ Abweichung \\
Diese Abweichung liegt innerhalb der maximal erlaubten Abweichung von 1 Prozent.
\newpage
\section{Kräfte}
In allen Schaltstellungen wirkt beim Stirnfräsen ein höheres Moment auf die Zahnräder als beim Umfangsfräsen. Deswegen wird im Folgenden nur mit den Werten der Torsionsmomente vom Stirnfräsen gerechnet. 
\begin{itemize}
\item Stirnradpaarungen: \\
Um die Zahnräder richtig auszulegen, werden die Kräfte für die jeweils höchstbelastete Stellung berechnet, in der das Zahnradpaar am Leistungsfluss beteiligt ist. 
Die verwendeten Formeln stammen aus dem Skript KL III\ccite{bib:poll:kl3} S. 7
\begin{align*}
	&F_t = \frac{2\x T}{d} \\
	&F_r= \frac{F_t \x \tan{\alpha_n}}{\cos(\beta)} \\
	&F_a = F_t \x \tan(\beta) \\
\end{align*}
\begin{align*}
&\textbf{Z1, Gang 4:} \\
	&F_{t,1} = \frac{2\x T_{\mathrm{I},4,S}}{d_1} = \frac{2 \x 95,29 \text{ Nm}}{0,0724 \text{ m}} = 2632,32 \text{ N}\\ 
	&F_{r,1} = \frac{2632,32 \text{ N} \x \tan{(20^\circ)}}{\cos(20^\circ)} = 1019,57\text{ N}\\ 
	&F_{a,1} =2632,32 \x \tan(20^\circ) = 958,1 \text{ N}\\
&\textbf{Z2, Gang 2:} \\
	&F_{t,2} = \frac{2\x T_{\mathrm{II},2,S}}{d_2} = \frac{2 \x 48,96 \text{ Nm}}{0,2171 \text{ m}} =451 \text{ N}\\ 
	&F_{r,2} = \frac{451 \text{ N} \x \tan{(20^\circ)}}{\cos(20^\circ)} = 174,69\text{ N}\\ 
	&F_{a,2} = 451 \x \tan(20^\circ) = 164,15 \text{ N}\\	
&\textbf{Z3, Gang 4:} \\
	&F_{t,3} = \frac{2\x T_{\mathrm{III},4,S}}{d_3} = \frac{2 \x 285,87 \text{ Nm}}{0,271 \text{ m}} = 2109,74 \text{ N}\\ 
	&F_{r,3} = \frac{2109,74 \text{ N} \x \tan{(20^\circ)}}{\cos(20^\circ)} = 817,16\text{ N}\\ 
	&F_{a,3} = 2109,74 \x \tan(20^\circ) = 767,88 \text{ N}\\
&\textbf{Z4, Gang 4:} \\
	&F_{t,4} = \frac{2\x T_{\mathrm{II},4,S}}{d_4} = \frac{2 \x 71,47 \text{ Nm}}{0,0894 \text{ m}} = 1598,88 \text{ N}\\ 
	&F_{r,4} = \frac{1598,88 \text{ N} \x \tan{(20^\circ)}}{\cos(20^\circ)} = 619,29\text{ N}\\ 
	&F_{a,4} = 1598,88 \x \tan(20^\circ) = 581,94 \text{ N}\\
&\textbf{Z5, Gang 4:} \\
	&F_{t,5} = \frac{2\x T_{\mathrm{IV},4,S}}{d_5} = \frac{2 \x 268,99 \text{ Nm}}{0,3405 \text{ m}} = 1579,97 \text{ N}\\ 
	&F_{r,5} = \frac{1579,97 \text{ N} \x \tan{(20^\circ)}}{\cos(20^\circ)} = 611,97\text{ N}\\ 
	&F_{a,5} = 1579,97 \x \tan(20^\circ) = 575,06 \text{ N}\\
&\textbf{Z6, Gang 3:} \\
	&F_{t,6} = \frac{2\x T_{\mathrm{III},3,S}}{d_6} = \frac{2 \x 121,63 \text{ Nm}}{0,149 \text{ m}} = 1632,62 \text{ N}\\ 
	&F_{r,6} = \frac{1632,62 \text{ N} \x \tan{(20^\circ)}}{\cos(20^\circ)} = 632,36\text{ N}\\ 
	&F_{a,6} = 1632,62 \x \tan(20^\circ) = 594,23 \text{ N}\\
&\textbf{Z7, Gang 3:} \\
	&F_{t,7} = \frac{2\x T_{\mathrm{IV},3,S}}{d_7} = \frac{2 \x 229,28 \text{ Nm}}{0,2809 \text{ m}} = 1632,47 \text{ N}\\ 
	&F_{r,7} = \frac{1632,47 \text{ N} \x \tan{(20^\circ)}}{\cos(20^\circ)} = 632,3\text{ N}\\ 
	&F_{a,7} = 1632,47 \x \tan(20^\circ) = 594,17 \text{ N}\\
\end{align*}
\item Planetengetriebe: \\
Auf die Zahnräder des Planetengetriebes wirken im vierten Gang die höchsten Torsionsmomente. Da vier Planetenräder mit dem Sonnenrad im Eingriff sind, wirkt auf jedes Planetenrad nur 1/4 der zur Momentenübertragung benötigten Kraft.
\begin{align*}
	&F_{t,8} = F_{t,9}= F_{t,10}=\frac{2\x T_{\mathrm{II},4,S}}{d_8 \x 4} = \frac{2 \x 71,47 \text{ Nm}}{0,112 \text{ m} \x 4} = 319,1\text{ N}\\ 
	&F_{r,8} = F_{r,9}= F_{r,10}=319,1 \text{ N} \x \tan{(20^\circ)} = 116,13\text{ N}\\ 
	&F_{a,8} = F_{a,9}= F_{a,10}=0 \text{ N}\\
\end{align*}
\item Kegelräder:\\
Formel aus Roloff/Matek\ccite{bib:roloffMatek:maschinenelemente} Seite 795 
\begin{align*}
	&d_{m,11} = d_{m,12}= d_{11} - b_{11} \x \sin{\delta_{11}} = 120 \text{ mm} - 28\text{mm} \x \sin{(45^\circ)} = 100,2 \text{ mm}\\ 
	&F_{tm,11} = F_{tm,12}= \frac{2 \x T_{\mathrm{V}}}{d_{m,11}} = \frac{2 \x 268,99 \text{ Nm} } {0,1002 \text{ m}} = 5369,06 \text{ N}\\
	&F_{r,11} = F_{r,12} = F_{tm,11} \x \tan{\alpha} \x \cos{\delta_{11}} =1381,81 \text{ N}\\ 
	&F_{a,11} = F_{a,12} =F_{tm,11}\x \tan{\alpha} \x \sin{\delta_{11}} = 1381,81 \text{ N} \\ 
\end{align*}
\end{itemize}