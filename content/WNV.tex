\chapter{Schwachstellenberechnung}
\section{Berechnung Welle-Nabe-Verbindungen}
In diesem Kapitel wird die maximale Flächenpressung für jede Passfederverbindung berechnet. Dieses wird anschließend mit der durch den Werkstoff vorgegebenen zulässigen Pressung verglichen. \\
Da hier die maximale Pressung bestimmt wird, muss für die Berechnung der 1,5 fache Wert des maximalen Momentes verwendet werden, da die Kupplung erst bei diesem Moment durch rutscht. \\ 
\begin{itemize}
	\item Zulässige Flächenpressung \hfill  (Skript KL IV\ccite{bib:poll:kl4} S. 56)
	\begin{align*}
		\text{Werkstoff C45: } &R_{e,min} = 430 \frac{\text{N}}{\text{mm}^2} \text{ bei } d\le 40 \text{ mm} \\
		 &R_{e,min} = 370 \frac{\text{N}}{\text{mm}^2} \text{ bei } d> 40 \text{ mm} \\
		 &p_{zul} = 0,9 \x R_{e,min}  \\
		 &p_{zul} = 387 \frac{\text{N}}{\text{mm}^2} \text{ für } d\le 40 \text{ mm} \\
		  &p_{zul} = 330 \frac{\text{N}}{\text{mm}^2} \text{ für } d> 40 \text{ mm} \\
	\end{align*}
	\item maximale Flächenpressung \hfill (Skript KL IV\ccite{bib:poll:kl4} S. 55 - 56)
	\begin{align*}
		&p_m = \frac{2 \x c_B \x T}{d \x h' \x l' \x n \x \phi} && \\
		&\text{für alle Passfedern gilt: } &n=1 & \\
		& &\phi = 1& \\
		& &c_B = 1 &\text{ (leichte Stöße, Tabelle 3.2 aus Skript KL IV)}\\
		& l' = l - b \text{ wenn } l'\le 1,2 \x d &&\implies \text{sonst setze } l' = 1,2 \x d \\
		&h' = 0,45 \x h &&\\
	\end{align*}
	Als Beispiel wird Passfeder 1 (Positionsnummer 19) auf Welle II berechnet:
	\begin{align*}
		&M= 1,5 \x M_{\mathrm{II}} = 121,98 \text{ Nm} \\
		&d= 40 \text{ mm} \\
		&l= 28 \text{ mm} \\
		&h= 8 \text{ mm} \\
		&b= 12 \text{ mm} \\
		&l'= 16 \text{ mm} \\
		&h'= 3,6 \text{ mm} \\ \\
		&\implies p_m = \frac{2 \x 1,1 \x 121980 \text{ Nmm}}{40 \text{ mm} \x 3,6 \text{ mm} \x 16 \text{ mm} \x 1 \x 1}  = 116,47 \frac{\text{N}}{\text{mm}^2}\\
		& p_m < p_{zul} = 387 \frac{\text{N}}{\text{mm}^2} \\
	\end{align*}
	Als weiteres Beispiel wird die Passfeder 10 (Positionsnummer 94) an der Schaltvorrichtung berechnet, da hier je nachdem, welche Zahnräder im Eingriff sind, unterschiedliche Längen der Passfeder beansprucht werden:
	\begin{align*}
	&d= 70 \text{ mm} &\\
	&h= 12 \text{ mm} &\\
	&b= 20 \text{ mm} &\\
	&h'= 5,4 \text{ mm}& \\ \\
	&\text{Gang 1:} &&l_1 = 63 \text{ mm, } l_1' = 43 \text{ mm}\\
	& && M_1 = 1,5 \x  M_{\mathrm{IV},1} = 0,96 \text{ Nm} \\
	& &&\implies p_{m,1} = 0,13\frac{\text{N}}{\text{mm}^2} \\ \\
	&\text{Gang 2:} &&l_2 = 30,8\text{ mm, } l_2' = 10,8 \text{ mm}\\
	& && M_2 = 1,5 \x  M_{\mathrm{IV},2} = 1,37 \text{ Nm} \\
	& &&\implies p_{m,2} = 0,74\frac{\text{N}}{\text{mm}^2} \\
	\end{align*}
\end{itemize}
\newpage
Alle weiteren Werte der Passfederberechnung sind in der folgenden Tabelle dargestellt. Alle Längen werden in der Einheit Millimeter angegeben. \\ \\
\begin{tabular}{|c|c|c|c|c|c|c|c|c|c|c|c|}\hline
	 Passfeder & Pos. Nr. &Welle & d & l & b & h & l' & h' & T(Nm) & $p_m(\frac{\text{N}}{\text{mm}^2})$ \\ \hline \hline
	 1, bei $z_9$ & 19 & $\mathrm{II}$ &40 & 28 & 12 & 8 & 16 & 3,6 & 121,98 & 116,47 \\ \hline
	 2, bei $z_8$ & 19 &$\mathrm{I}$ & 40 & 28 & 12 & 8 & 16 & 3,6 & 90,26 & 86,19 \\ \hline
	 3, bei $z_6$ & 25 &$\mathrm{I}$ & 50 & 40 & 14 & 9 & 26 & 4,05 & 90,26 & 37,72 \\ \hline
	4, bei $z_4$ & 26 &$\mathrm{I}$ &  50 & 36 & 14 & 9 & 22 & 4,05 & 90,26 & 44,57 \\ \hline
	5, bei $z_1$ & 19 & $\mathrm{I}$ & 40 & 28 & 12 & 8 & 16 & 3,6 & 90,26 & 86,19 \\ \hline
	 6, bei Kupplung & 35 &$\mathrm{I}$ & 20 & 20 & 6 & 6 & 14 & 2,7 & 90,26 & 262,66 \\ \hline
	 7, bei $z_{10}$ & 105 &$\mathrm{IV}$ & 30 & 70 & 10 & 8 & 36 & 3,6 & 1,37 & 0,78 \\ \hline
	 8, bei Schaltung & 94 &$\mathrm{IV}$ & 40 & 140 & 12 & 8 & 48 & 3,6 & 1,37 & 0,44 \\ \hline
	 9, bei $z_{11}$ & 103 & $\mathrm{V}$ &35 & 32 & 10 & 8 & 22 & 3,6 & 13,7 & 10,87 \\ \hline
	 10, bei Schaltung & 91 &Gang 1 & 70 & 63 & 20 & 12 & 43 & 5,4 & 0,96 & 0,13 \\ \hline
	 10, bei Schaltung & 91 & Gang 2 &70 & 30,8 & 20 & 12 & 10,8 & 5,4 & 1,37 & 0,74 \\ \hline
\end{tabular} \\
\vspace{.5cm}
\\ Da bei allen Passfedern gilt $p_m < p_zul$ halten die Passfedern der Belastung stand.