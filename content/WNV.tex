\chapter{Schwachstellenberechnung}
\section{Berechnung Welle-Nabe-Verbindungen}
In diesem Kapitel wird die maximale Flächenpressung für jede Passfederverbindung berechnet. Dieses wird anschließend mit der durch den Werkstoff vorgegebenen zulässigen Pressung verglichen. Als Sicherheitsfaktor wird S=1,5 gewählt.\\
\begin{itemize}
	\item Zulässige Flächenpressung \hfill  (Skript KL IV\ccite{bib:poll:kl4} S. 56)
	\begin{align*}
		\text{Werkstoff C45: } &R_{e,min} = 430 \frac{\text{N}}{\text{mm}^2} \text{ bei } d\le 40 \text{ mm} \\
		 &R_{e,min} = 370 \frac{\text{N}}{\text{mm}^2} \text{ bei } d> 40 \text{ mm} \\
		 &p_{zul} = 0,9 \x R_{e,min}  \\
		 &p_{zul} = 387 \frac{\text{N}}{\text{mm}^2} \text{ für } d\le 40 \text{ mm} \\
		  &p_{zul} = 330 \frac{\text{N}}{\text{mm}^2} \text{ für } d> 40 \text{ mm} \\
	\end{align*}
	\item maximale Flächenpressung \hfill (Skript KL IV\ccite{bib:poll:kl4} S. 55 - 56)
	\begin{align*}
		&p_m = \frac{2 \x c_B \x T}{d \x h' \x l' \x n \x \phi} && \\
		&\text{für alle Passfedern gilt: } &n=1 & \\
		& &\phi = 1& \\
		& &c_B = 1 &\text{ (leichte Stöße, Tabelle 3.2 aus Skript KL IV)}\\
		& l' = l - b \text{ wenn } l'\le 1,2 \x d &&\implies \text{sonst setze } l' = 1,2 \x d \\
		&h' = 0,45 \x h &&\\
	\end{align*}
	Als Beispiel wird Passfeder 2 (Positionsnummer 39) auf Welle II berechnet:
	\begin{align*}
		&M= 1,5 \x T_{\mathord{\mathrm{II}},4,S} = 1,5 \x  71,47\text{ Nm} = 107,21 \text{ Nm} \\
		&d= 35 \text{ mm} \\
		&l= 22 \text{ mm} \\
		&h= 8 \text{ mm} \\
		&b= 10 \text{ mm} \\
		&l'= 12 \text{ mm} \\
		&h'= 3,6 \text{ mm} \\ \\
		&\implies p_m = \frac{2 \x 1,1 \x 107210 \text{ Nmm}}{35 \text{ mm} \x 3,6 \text{ mm} \x 12 \text{ mm} \x 1 \x 1}  = 155,99 \frac{\text{N}}{\text{mm}^2}\\
		& p_m < p_{zul} = 387 \frac{\text{N}}{\text{mm}^2} \\
	\end{align*}
	Als weiteres Beispiel wird die Passfeder 1 (Positionsnummer 15) auf Welle I berechnet, da hier je nachdem, welche Zahnräder im Eingriff sind, unterschiedliche Längen der Passfeder beansprucht werden:
	\begin{align*}
	&d= 30 \text{ mm} &\\
	&h= 7 \text{ mm} &\\
	&b= 8 \text{ mm} &\\
	&h'= 3,15 \text{ mm}& \\ \\
	&\text{Z1/Z2:} &&l_1 = 40 \text{ mm, } l_1' = 32 \text{ mm}\\
	& && M_1 = 1,5 \x T_{\mathord{\mathrm{I}},2,S} = 1,5 \x 16,32 \text{ Nm} =24,48 \text{ Nm} \\
	& &&\implies p_{m,1} = 17,8\frac{\text{N}}{\text{mm}^2} \\ \\
	&\text{Z1/Z3:} &&l_2 = 54\text{ mm, } l_2' = 46 \text{ mm}\\
	& && M_2 = 1,5 \x T_{\mathord{\mathrm{I}},4,S} = 1,5 \x 95,29 \text{ Nm} = 142,94 \text{ Nm}\\
	& &&\implies p_{m,2} = 72,3\frac{\text{N}}{\text{mm}^2} \\
	\end{align*}
\end{itemize}
\newpage
Alle weiteren Werte der Passfederberechnung sind in der folgenden Tabelle dargestellt. Alle Längen werden in der Einheit Millimeter angegeben. \\ \\
\begin{tabular}{|c|c|c|c|c|c|c|c|c|c|c|c|}\hline
	 Passfeder & Pos. Nr. &Welle & d & l & b & h & l' & h' & T(Nm) & $p_m(\frac{\text{N}}{\text{mm}^2})$ \\ \hline \hline
	 1, bei $Z_1$(Eingriff Z2) & 19 & $\mathrm{I}$ &30 & 40 & 8 & 7 & 32 & 3,15 & 24,28 & 17,8 \\ \hline
	 1, bei $Z_1$(Eingriff Z2) & 19 & $\mathrm{I}$ &30 & 54 & 8 & 7 & 46 & 3,15 & 142,94 & 72,3 \\ \hline
	 2, bei $Z_4$ & 39 &$\mathrm{II}$ & 35 & 22 & 10 & 8 & 12 & 3,6 & 107,21 & 155,99 \\ \hline
	 3, bei $Z_2$ & 102 &$\mathrm{II}$ & 42 & 32 & 12 & 8 & 20 & 3,6 & 107,21 & 78 \\ \hline
	4, bei $Z_8$ & 39 &$\mathrm{II}$ & 35 & 22 & 10 & 8 & 12 & 3,6 & 107,21 & 155,99 \\ \hline
	5, bei Kupplung & 127 & $\mathrm{IV}$ & 55 & 28 & 16 & 10 & 66 & 4,5 & 268,99 & 36,23 \\ \hline
	 6, bei Kupplung & 118 &$\mathrm{IV}$ & 85 & 20 & 12 & 8 & 8 & 3,6 & 268,99 & 241,74 \\ \hline
	 7, bei $Z_{11}$ & 75 &$\mathrm{V}$ & 40 & 28 & 12 & 8 & 16 & 3,6 & 268,99 & 256,85 \\ \hline
	 8, bei $Z_{12}$ & 94 &$\mathrm{VI}$ & 40 & 28 & 12 & 8 & 16 & 3,6 & 268,99 & 256,85 \\ \hline
\end{tabular} \\
\vspace{.5cm}
\\ Da bei allen Passfedern gilt $p_m < p_{zul}$ halten die Passfedern der Belastung stand.