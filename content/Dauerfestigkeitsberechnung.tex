\chapter{Dauerfestigkeitsberechnung für Welle I}
In diesem Kapitel werden die beiden am meisten gefährdeten Querschnitte der Welle I auf Dauerfestigkeit und bleibende Verformung untersucht. Der Sicherheitsfaktor sollte dabei nach DIN 743 mindestens $S_{min} = 1,2$ betragen.
\section{Werkstoffkennwerte}
Als Werkstoff wurde der Vergütungsstahl 34CrMo4 gewählt. Die folgenden Festigkeitswerte stammen aus der Tabelle A.4 der DIN 743 - 3.
\begin{align*}
	&R_{p0,2} = 800 \frac{\text{ N}}{\text{ mm}^2} = \sigma_S \\
	&R_{m} = 1000 \frac{\text{ N}}{\text{ mm}^2} = \sigma_B \\
	&\sigma_{zdW} = 400 \frac{\text{ N}}{\text{ mm}^2} \\
	&\sigma_{bW} = 500 \frac{\text{ N}}{\text{ mm}^2} \\
	&\tau_{tW} = 300 \frac{\text{ N}}{\text{ mm}^2} 
\end{align*}

\section{Passfedernutberechnung}
Der Durchmesser des gefährdeten Absatzes ist $d=40$ mm. Die Sicherheit der Passfederverbindung wird bei Umlaufbiegung und statischer Torsion berechnet. Da hier die Dauerfestigkeit betrachtet wird, wird mit dem im Betrieb wirksamen Torsionsmoment gerechnet anstatt den Sicherheitsfaktor der Kupplung mit einzubeziehen. \\
Als Bezugsdurchmesser wird $d_{BK} = 40$ mm verwendet.
\subsubsection{Wirkende Spannungen (nach DIN 743 - 1 Tabelle 1)}
\begin{align*}
	&\sigma_{zda} = \frac{F_{zda}}{A} = \frac{N}{\frac{\pi \x d^2}{4}} = \frac{2318,49 \text{ N}}{1256,6 \text{ mm}^2} = 1,85 \frac{\text{ N}}{\text{ mm}^2} \\
	&\sigma_{ba} = \frac{M_{ba}}{A} = \frac{|M_{b,res}(z=409 \text{ mm}) |}{\frac{\pi \x d^3}{32}} = \frac{235730 \text{ Nmm}}{6283,19 \text{ mm}^3} = 37,52 \frac{\text{ N}}{\text{ mm}^2} \\
	&\tau_{ta} = \frac{M_{ta}}{A} = \frac{M_{\mathrm{I}}}{\frac{\pi \x d^3}{16}} = \frac{60170 \text{ Nmm}}{12566,37 \text{ mm}^3} = 4,79 \frac{\text{ N}}{\text{ mm}^2} 
\end{align*}

\subsubsection{Gesamteinflussfaktor für Biegung}
Die folgenden Formeln stammen, soweit nichts anderes vermerkt, aus der DIN 743 - 2.
\begin{itemize}
	\item Technologischer Größeneinflussfaktor $K_1 (d_{eff})$ 
	\begin{align*}
		&\text{mit } d_{eff} = 40 \text{ mm und }d_B = 16 \text{ mm} \\ 
		&\text{Streckgrenze Vergütungsstahl: }K_{1,Re}(d_{eff}) = 1 - 0,34 \lg \left( \frac{d_{eff}}{d_B} \right) = 0,865  &(14) \\
		&\text{Zugfestigkeit Vergütungsstahl: } K_{1,Rm}(d_{eff}) = 1 - 26 \lg \left( \frac{d_{eff}}{d_B} \right) = 0,897  &(12) 
	\end{align*}
	\item Kerbwirkungszahl $\beta_{\sigma} (d_{BK})$
	\begin{align*}
		&\sigma_B (d) = \sigma_B \x K_{1,Rm} (d_{eff} = 40 \text{ mm}) = 897 \frac{\text{ N}}{\text{ mm}^2} \\
		&\implies \beta_{\sigma}(d_{BK})	= 2,89  & \text{ (Tabelle 1)} 
	\end{align*}
	\item  Geometrischer Größeneinflussfaktor $K_3 (d)$ \hfill (17)
	\begin{align*}
		K_3(d) &= 1 - 0,2 \x \lg ( \beta_{\sigma} (d_{BK})) \x \frac{\lg (d / 7,5 \text{ mm})}{lg 20} \\
		&= 0,949 \\
		K_3(d_{BK}) &= 1 - 0,2 \x \lg ( \beta_{\sigma} (d_{BK})) \x \frac{\lg (d_{BK} / 7,5 \text{ mm})}{lg 20} \\
		&= 0,949 
	\end{align*}
	\item Kerbwirkungszahl $\beta_{\sigma}$ für d \hfill (3)
	\[
		\beta_{\sigma} = \beta_{\sigma} (d_{BK}) \x \frac{K_3 (d_{BK})}{K_3 (d)} = 2,89
	\]
	\item Geometrischer Größeneinflussfaktor $K_2 (d)$ \hfill (16)
	\[
		K_2 (d) = 1 -0,2 \x  \frac{\lg (d / 7,5 \text{ mm})}{lg 20}  = 0,888
	\]
	\item Einflussfaktor der Oberflächenrauheit (aus KL III \ccite{bib:poll:kl3} Abb. 6.4.9)
	\[
		K_{F \sigma} = 0,88 \text{ mit } R_z = 6,3
	\]
	\item Einflussfaktor der Oberflächenverfestigung (aus KL III \ccite{bib:poll:kl3} Seite 172)
	\[
	K_{V} = 1
	\]
	\item  Gesamteinflussfaktor $K_{\sigma}$ aus DIN 743 - 1 \hfill (8)
	\[
		K_{\sigma} = \left( \frac{\beta_{\sigma}}{K_2 (d)} + \frac{1}{K_{F\sigma}} -1 \right) \x \frac{1}{K_V} = 3,39 
	\]
\end{itemize}
\subsubsection{Vorhandene Sicherheitszahl für Dauerfestigkeitsnachweis nach Belastungsfall 1 }
verwendete Formeln aus der DIN 743 - 1
\begin{itemize}
	\item Vergleichsmittelspannung \hfill (23)
	\begin{align*}
		&\sigma_{mv}= \sqrt{\sigma_{bm}^2+ 3 \x \tau_{tm}^2 }  \\
		&\sigma_{bm}= 0 \text{ (da Umlaufbiegung vorliegt)} \\
		&\tau_{tm} = \tau_{ta} = 4,79 \frac{\text{ N}}{\text{ mm}^2} \\
		&\implies \sigma_{mv} = 8,3 \frac{\text{ N}}{\text{ mm}^2} 
	\end{align*}
	\item Bauteilwecheselfestigkeit $\sigma_{bWK}$ \hfill (6)
	\begin{align*}
		\sigma_{bWK}&= \frac{\sigma_{bW} \x K_{1,Rm} (d_{eff})}{K_{\sigma}}  \\
		&=  \frac{500 \frac{\text{ N}}{\text{ mm}^2}\x 0,897}{3,39} = 132,3 \frac{\text{ N}}{\text{ mm}^2}
	\end{align*}
	\item Einflussfaktor der Mittelspannungsempfindlichkeit $\Psi_{b \sigma K}$ \hfill (21)
	\begin{align*}
		\Psi_{b \sigma K}&= \frac{\sigma_{bWK}}{2 \x  K_{1,Rm} (d_{eff}) \x \sigma_B (d_B) -\sigma_{bWK}}  \\
		&=  \frac{132,3 \frac{\text{ N}}{\text{ mm}^2}}{2 \x 0,897 \x 1000\frac{\text{ N}}{\text{ mm}^2} - 132,3 \frac{\text{ N}}{\text{ mm}^2}} = 0,08 
	\end{align*}
	\item Spannungsamplitude der Bauteildauerfestigkeit $\sigma_{bADK}$ \hfill (11) 
	\begin{align*}
		\sigma_{bADK} &= \sigma_{bWK} - \Psi_{b \sigma K} \x \sigma_{mv} \\
		&= 132,3 \frac{\text{ N}}{\text{ mm}^2} - 0,08 \x 8,3 \frac{\text{ N}}{\text{ mm}^2} = 131,6 \frac{\text{ N}}{\text{ mm}^2} 
	\end{align*}
	\item vorhandene Sicherheitszahl S \hfill (3)
	\[
		S= \frac{\sigma_{bADK}}{\sigma_{ba}} = \frac{131,6 \frac{\text{ N}}{\text{ mm}^2}}{37,52 \frac{\text{ N}}{\text{ mm}^2} } = 3,5
	\]
\end{itemize}
\subsubsection{Vorhandene Sicherheitszahl S für Nachweis gegen Überschreiten der Fließgrenze}
verwendete Formeln aus der DIN 743 - 1
\begin{itemize}
	\item Statische Stützwirkung $K_{2F}$ für Biegung und Torsion nach Tabelle 3
	\begin{align*}
		&K_{2F \sigma} = 1,2 \\
		&K_{2F \tau} = 1,2 
	\end{align*}
	\item Erhöhungsfaktor der Fließgrenze $\gamma_{F\sigma}$ nach Tabelle 2
	\begin{align*}
		&\gamma_{F\sigma} = 1,1 \text{ (Beanspruchung Biegung bei } \beta_{\sigma} = 2,86 \text{)} \\
		&\gamma_{F\tau} = 1 
	\end{align*}
	\item Bauteilfließgrenze
	\begin{align*}
		\sigma_{bFK} &= K_{1,Re} (d_{eff}) \x K_{2F} \x \gamma_{F\sigma} \x \sigma_S (d_B) & (31)\\
		&= 0,865 \x 1,2 \x 1,1 \x 800 \frac{\text{ N}}{\text{ mm}^2} \\
		&= 913,44 \frac{\text{ N}}{\text{ mm}^2}\\
		\tau_{tFK} &= K_{1,Re} (d_{eff}) \x K_{2F} \x \gamma_{F\tau} \x \sigma_S (d_B) / \sqrt{3} & (32)\\
		&= 0,865 \x 1,2 \x 1 \x 800 \frac{\text{ N}}{\text{ mm}^2} / \sqrt{3}\\
		&= 479,43 \frac{\text{ N}}{\text{ mm}^2}
	\end{align*}
	\item Vorhandene Sicherheitszahl S 
	Hier wird eine stoßartige Belastung der Passfeder angenommen, da es aufgrund der wirkenden Schnittkräfte an Welle II zu leichten Stößen kommen kann.
	\begin{align*}
		&\sigma_{bmax} = 1,5 \x \sigma_{ba} = 56,28 \frac{\text{ N}}{\text{ mm}^2} \\
		&\tau_{tmax} = 1,5 \x \tau_{m} = 7,19 \frac{\text{ N}}{\text{ mm}^2} \\
		&S = \frac{1}{\sqrt{\left( \frac{\sigma_{bmax}}{\sigma_{bFK}} \right)^2 +\left( \frac{\tau_{tmax}}{\tau_{tFK}} \right)^2 }} & (25)\\
		&= 15,8 
	\end{align*}
\end{itemize}
Die Berechnung zeigt, dass die Passfedernut gegen Dauerbruch und bleibende Verformung ausreichend ausgelegt wurde und somit der Belastung stand hält.
\newpage
\section{Freistichberechnung}
\subsubsection{Abmessungen:}
\begin{align*}
	&D= 47 \text{ mm} & \frac{r}{t} = 0,21 \\
	&d= 39,4 \text{ mm} & \frac{r}{d} = 0,02 \\
	& r= 0,8 \text{ mm} & \frac{d}{D} = 0,838 \\
	& t= \frac{D-d}{2} = 3,8 \text{ mm} 
\end{align*}
\subsubsection{Wirkende Spannungen (nach DIN 743 - 1 Tabelle 1)}
\begin{align*}
	&\sigma_{zda} = \frac{F_{zda}}{A} = \frac{N}{\frac{\pi \x d^2}{4}} = \frac{581,94 \text{ N}}{1219,22 \text{ mm}^2} = 0,48 \frac{\text{ N}}{\text{ mm}^2} \\
	&\sigma_{ba} = \frac{M_{ba}}{A} = \frac{|M_{b,res}(x=345 \text{ mm}) |}{\frac{\pi \x d^3}{32}} = \frac{541240 \text{ Nmm}}{6004,66 \text{ mm}^3} = 90,1 \frac{\text{ N}}{\text{ mm}^2} \\
	&\tau_{ta} = \frac{M_{ta}}{A} = \frac{M_{\mathrm{II}}}{\frac{\pi \x d^3}{16}} = \frac{71500 \text{ Nmm}}{12009,32 \text{ mm}^3} = 5,95 \frac{\text{ N}}{\text{ mm}^2} 
\end{align*}
\subsubsection{Gesamteinflussfaktor}
Die folgenden Formeln stammen, soweit nichts anderes vermerkt, aus der DIN 743 - 2.
\begin{itemize}
	\item Bezogenes Spannungsgefälle $G'$ \hfill (Tabelle 2)
	\begin{align*}
		\text{Hilfsgröße } \Phi &= \frac{1}{4 \x \sqrt{\frac{t}{r}} +2} = 0,09 \\
		G'_{ZD} &= \frac{2,3 \x (1+\Phi)}{r} = 3,13 \frac{1}{\text{ mm}} \\
		G'_{B}&= G'_{ZD}  =3,13 \frac{1}{\text{ mm}} \\
		G'_{T}&= \frac{1,15}{r} = 1,44\frac{1}{\text{ mm}} 
	\end{align*}
	\item Technologischer Größeneinflussfaktor $K_1 (d_{eff})$ 
	\begin{align*}
		&\text{zur Vereinfachung wird angenommen } d_{eff} = D =47 \text{ mm und } d_B = 16 \text{ mm} \\ 
		&\text{Streckgrenze Vergütungsstahl: }K_{1,Re}(d_{eff}) = 1 - 0,34 \lg \left( \frac{d_{eff}}{d_B} \right) = 0,84  &(14) \\
		&\text{Zugfestigkeit Vergütungsstahl: } K_{1,Rm}(d_{eff}) = 1 - 26 \lg \left( \frac{d_{eff}}{d_B} \right) = 0,88  &(12) 
	\end{align*}
	\item Formzahl $\alpha$
	\begin{align*}
		&\alpha_{\sigma ZD} = 1 + \frac{1}{\sqrt{0,62 \x \frac{r}{t} + 7 \x \frac{r}{d} (1+2 \x \frac{r}{d})^2}} = 2,88 & \text{(Bild 8)} \\
		&\alpha_{\sigma B} = 1 + \frac{1}{\sqrt{0,62 \x \frac{r}{t} + 11,6 \x \frac{r}{d} (1+2 \x \frac{r}{d})^2 + 0,2 \x \left( \frac{r}{t} \right) ^3 \x \frac{d}{D}}} = 2,62 & \text{(Bild 9)} \\
		&\alpha_{\tau} = 1 + \frac{1}{\sqrt{3,4 \x \frac{r}{t} + 38 \x \frac{r}{d} (1+2 \x \frac{r}{d})^2 + \left( \frac{r}{t} \right) ^2 \x \frac{d}{D}}} = 1,8  & \text{(Bild 10)} 
	\end{align*}
	\item Stützziffer n \hfill (5)
	\begin{align*}
		n&= 1 + \sqrt{G' \x \text{ mm}} \x 10 ^{-\left( 0,33 + \frac{\sigma_S (d)}{712\frac{\text{ N}}{\text{ mm}^2}} \right)} \\
		\text{mit } \sigma_S (d) &= K_{1,Re} (d_{eff}) \x \sigma_S (d_B) \\
		&= 0,84 \x 800 \frac{\text{ N}}{\text{ mm}^2} = 672 \frac{\text{ N}}{\text{ mm}^2} \\
		n_{ZD} &= n_B = 1 + \sqrt{3,13} \x 10 ^{-\left( 0,33 + \frac{672}{712} \right)} \\
		&=1,09 \\
		n_{\tau} &= 1 + \sqrt{1,44} \x 10 ^{-\left( 0,33 + \frac{672}{712} \right)} \\
		&= 1,06 
	\end{align*}
	Formel für $\sigma_S$ aus Skript KL III \ccite{bib:poll:kl3} S.168
	\item Kerbwirkungszahl $\beta$ \hfill (4)
	\begin{align*}
		&\beta_{\sigma ZD} = \frac{\alpha_{\sigma ZD}}{n_{ZD}} = 2,64 \\
		&\beta_{\sigma B} = \frac{\alpha_{\sigma B}}{n_{B}} = 2,4 \\
		&\beta_{ \tau} = \frac{\alpha_{\sigma \tau}}{n_{\tau}} = 1,7 
	\end{align*}
	\item Geometrischer Größeneinflussfaktor $K_2 (d)$ 
	\begin{align*}
		&\text{Zug/Druck: }K_{2ZD} (d) = 1 & (15) \\
		&\text{Biegung/Torsion: } K_{2B,\tau} (d) = 1 -0,2 \x  \frac{\lg (d / 7,5 \text{ mm})}{lg 20}  = 0,89 &(16) 
	\end{align*}
	\item Einflussfaktor der Oberflächenrauheit $K_{F\sigma, \tau}$ (Skript KL III \ccite{bib:poll:kl3} S.171)
	\begin{align*}
		\text{Zug/Druck \& Biegung: } K_{F\sigma} &= 1 - 0,22 \x \lg \left( \frac{Rz}{\mu m}\right) \x ( \lg \left( \frac{\sigma_B (d)}{20}\frac{\text{ N}}{\text{ mm}^2}\right) -1 ) &(18) \\
		\text{mit } \sigma_B (d) &= K_{1,Re} (d_{eff}) \x \sigma_B (d_B) \\
		&= 0,84 \x 1000\frac{\text{ N}}{\text{ mm}^2} = 840 \frac{\text{ N}}{\text{ mm}^2} \\
		\implies K_{F\sigma} &= 0,89 \text{ mit } R_z = 6,3 \mu m\\
		\text{Torsion: } K_{F\tau} &= 0,575 \x K_{F\sigma} +0,425 = 0,937 & (19) 
	\end{align*}
	\item Einflussfaktor der Oberflächenverfestigung (aus KL III \ccite{bib:poll:kl3} Seite 172)
	\[
		K_{V} = 1
	\]
	\item Gesamteinflussfaktor $K_{\sigma}$ nach DIN 743 - 1
	\begin{align*}
		\text{Zug/Druck: } K_{\sigma ZD} &= \left( \frac{\beta_{\sigma ZD}}{K_{2ZD} (d)} + \frac{1}{K_{F\sigma}} -1\right) \x \frac{1}{K_V} & (8) \\
		&= \left( \frac{2,64}{1} + \frac{1}{0,89} -1\right) \x \frac{1}{1} \\
		&= 2,76 \\
		\text{Biegung: } K_{\sigma B} &= \left( \frac{\beta_{\sigma B}}{K_{2B}(d)} + \frac{1}{K_{F\sigma}} -1\right) \x \frac{1}{K_V} & (8) \\
		&= \left( \frac{2,4}{0,89} + \frac{1}{0,89} -1\right) \x \frac{1}{1} \\
		&= 2,82 \\
		\text{Torsion: } K_{\tau} &= \left( \frac{\beta_{\tau}}{K_{2\tau}(d)} + \frac{1}{K_{F\tau}} -1\right) \x \frac{1}{K_V} & (9) \\
		&= \left( \frac{1,7}{0,89} + \frac{1}{0,937} -1\right) \x \frac{1}{1} \\
		&= 1,98 
	\end{align*}
\end{itemize}
\subsubsection{Vorhandene Sicherheitszahl für Dauerfestigkeitsnachweis nach Belastungsfall 1 }
verwendete Formeln aus der DIN 743 - 1
\begin{itemize}
	\item Vergleichsmittelspannung 
	\begin{align*}
	\sigma_{mv}&= \sqrt{(\sigma_{zdm} +\sigma_{bm} )^2+ 3 \x \tau_{tm}^2 } & (23) \\
	&\sigma_{bm}= 0 \text{ (da Umlaufbiegung vorliegt)} \\
	&\sigma_{zdm} = \sigma_{zda} = 0,48\frac{\text{ N}}{\text{ mm}^2} \\
	&\tau_{tm} = \tau_{ta} = 5,95 \frac{\text{ N}}{\text{ mm}^2} \\
	\implies \sigma_{mv}&= \sqrt{(0,48 \frac{\text{ N}}{\text{ mm}^2} +0 )^2+ 3 \x (5,95 \frac{\text{ N}}{\text{ mm}^2})^2 }  \\
	&= 10,32 \frac{\text{ N}}{\text{ mm}^2} \\
	\tau_{mv} &= \frac{\sigma_{mv}}{\sqrt{3}} = 5,96 \frac{\text{ N}}{\text{ mm}^2} & (24)
	\end{align*}
	\item Bauteilwecheselfestigkeit $\sigma_{WK}$ 
	\begin{align*}
	\sigma_{zdWK}&= \frac{\sigma_{zdW} \x K_{1,Rm} (d_{eff})}{K_{\sigma ZD}}  & (5)\\
	&=  \frac{400 \frac{\text{ N}}{\text{ mm}^2}\x 0,88}{2,76} = 127,54 \frac{\text{ N}}{\text{ mm}^2}\\
	\sigma_{bWK}&= \frac{\sigma_{bW} \x K_{1,Rm} (d_{eff})}{K_{\sigma B}}  & (6 )\\
	&=  \frac{500 \frac{\text{ N}}{\text{ mm}^2}\x 0,88}{2,82} = 156 \frac{\text{ N}}{\text{ mm}^2}\\
	\tau_{tWK}&= \frac{\tau_{tW} \x K_{1,Rm} (d_{eff})}{K_{\tau}} &(7) \\
	&=  \frac{300 \frac{\text{ N}}{\text{ mm}^2}\x 0,88}{1,98} = 133,3 \frac{\text{ N}}{\text{ mm}^2}
	\end{align*}
	\item Einflussfaktor der Mittelspannungsempfindlichkeit $\Psi_{K}$ 
	\begin{align*}
	\Psi_{zd \sigma K}&= \frac{\sigma_{zdWK}}{2 \x  K_{1,Rm} (d_{eff}) \x \sigma_B (d_B) -\sigma_{zdWK}}  &(20)\\
	&=  \frac{127,54 \frac{\text{ N}}{\text{ mm}^2}}{2 \x 0,88 \x 1000\frac{\text{ N}}{\text{ mm}^2} - 127,54 \frac{\text{ N}}{\text{ mm}^2}} = 0,078 \\
	\Psi_{b \sigma K}&= \frac{\sigma_{bWK}}{2 \x  K_{1,Rm} (d_{eff}) \x \sigma_B (d_B) -\sigma_{bWK}} &(21) \\
	&=  \frac{156 \frac{\text{ N}}{\text{ mm}^2}}{2 \x 0,88 \x 1000\frac{\text{ N}}{\text{ mm}^2} - 156 \frac{\text{ N}}{\text{ mm}^2}} = 0,097 \\
	\Psi_{\tau K}&= \frac{\tau_{tWK}}{2 \x  K_{1,Rm} (d_{eff}) \x \sigma_B (d_B) -\tau_{tWK}} &(22) \\
	&=  \frac{133,3 \frac{\text{ N}}{\text{ mm}^2}}{2 \x 0,88 \x 1000\frac{\text{ N}}{\text{ mm}^2} - 133,3 \frac{\text{ N}}{\text{ mm}^2}} = 0,082 
	\end{align*}
	\item Spannungsamplitude der Bauteildauerfestigkeit
	\begin{align*}
	\sigma_{zdADK} &= \sigma_{zdWK} - \Psi_{zd \sigma K} \x \sigma_{mv} &(10) \\
	&= 127,54\frac{\text{ N}}{\text{ mm}^2} - 0,078 \x 10,32 \frac{\text{ N}}{\text{ mm}^2} = 126,74 \frac{\text{ N}}{\text{ mm}^2} \\
	\sigma_{bADK} &= \sigma_{bWK} - \Psi_{b \sigma K} \x \sigma_{mv} &(11) \\
	&= 156\frac{\text{ N}}{\text{ mm}^2} - 0,097 \x 10,32 \frac{\text{ N}}{\text{ mm}^2} = 155 \frac{\text{ N}}{\text{ mm}^2} \\
	\tau_{tADK} &= \tau_{tWK} - \Psi_{ \tau K} \x \tau_{mv} &(12) \\
	&= 133,3\frac{\text{ N}}{\text{ mm}^2} - 0,082 \x 5,96 \frac{\text{ N}}{\text{ mm}^2} = 132,8 \frac{\text{ N}}{\text{ mm}^2} 
	\end{align*}
	\item vorhandene Sicherheitszahl S 
	\begin{align*}
		&S= \frac{1}{\sqrt{\left( \frac{\sigma_{zda}}{\sigma_{zdADK}} +\frac{\sigma_{ba}}{\sigma_{bADK}} \right)^2 +\left( \frac{\tau_{ta}}{\tau_{tADK}} \right)^2 }} &(2) \\
		&S=  1,7 
	\end{align*}
\end{itemize}
\subsubsection{Vorhandene Sicherheitszahl S für Nachweis gegen Überschreiten der Fließgrenze}
verwendete Formeln aus der DIN 743 - 1
\begin{itemize}
	\item Statische Stützwirkung $K_{2F}$ nach Tabelle 3
	\begin{align*}
	&K_{2F \sigma zd} = 1 \\
	&K_{2F \sigma b} = 1,2 \\
	&K_{2F \tau} = 1,2 
	\end{align*}
	\item Erhöhungsfaktor der Fließgrenze $\gamma_{F}$ nach Tabelle 2
	\begin{align*}
	&\gamma_{F\sigma} = 1,1 \text{ (Für } \beta_{\sigma} = 2,0 \text{ bis } 3,0 \text{)} \\
	&\gamma_{F\tau} = 1 
	\end{align*}
	\item Bauteilfließgrenze
	\begin{align*}
	\sigma_{zd,bFK} &= K_{1,Re} (d_{eff}) \x K_{2F \sigma} \x \gamma_{F\sigma} \x \sigma_S (d_B) & (31) \\
	\sigma_{zdFK}&= 0,84 \x 1 \x 1,1 \x 800 \frac{\text{ N}}{\text{ mm}^2} \\
	&= 739,2 \frac{\text{ N}}{\text{ mm}^2}\\
	\sigma_{bFK}&= 0,84 \x 1,2 \x 1,1 \x 800 \frac{\text{ N}}{\text{ mm}^2} \\
	&= 887\frac{\text{ N}}{\text{ mm}^2}\\
	\tau_{tFK} &= K_{1,Re} (d_{eff}) \x K_{2F \tau} \x \gamma_{F\tau} \x \sigma_S (d_B) / \sqrt{3} &(32) \\
	&= 0,84 \x 1,2 \x 1 \x 800 \frac{\text{ N}}{\text{ mm}^2} / \sqrt{3}\\
	&= 465,58 \frac{\text{ N}}{\text{ mm}^2}
	\end{align*}
	\item Vorhandene Sicherheitszahl S \\
	Für diesen Fall wird mit den wirkenden Spannungsamplituden als maximale Spannungen gerechnet, da keine stoßartige Belastung angenommen wird. 
	\begin{align*}
	&S = \frac{1}{\sqrt{\left( \frac{\sigma_{zdmax}}{\sigma_{zdFK}}+\frac{\sigma_{bmax}}{\sigma_{bFK}} \right)^2 +\left( \frac{\tau_{tmax}}{\tau_{tFK}} \right)^2 }} & (25)\\
	&= 9,7 
	\end{align*}
\end{itemize}
Die Schwachstelle 2, also der Freistich, ist ebenfalls gegen Dauerbruch und plastische Verformung ausreichend ausgelegt.