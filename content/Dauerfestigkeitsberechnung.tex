\chapter{Dauerfestigkeitsberechnung für Welle I}
In diesem Kapitel werden die beiden am meisten gefährdeten Querschnitte der Welle I auf Dauerfestigkeit und bleibende Verformung untersucht. Der Sicherheitsfaktor sollte dabei nach DIN 743 mindestens $S_{min} = 1,2$ betragen.
\section{Werkstoffkennwerte}
Als Werkstoff wurde der Vergütungsstahl 34CrMo4 gewählt. Die folgenden Festigkeitswerte stammen aus der Tabelle A.4 der DIN 743 - 3.
\begin{align*}
	&R_{p0,2} = 800 \frac{\text{ N}}{\text{ mm}^2} = \sigma_S \\
	&R_{m} = 1000 \frac{\text{ N}}{\text{ mm}^2} = \sigma_B \\
	&\sigma_{zdW} = 400 \frac{\text{ N}}{\text{ mm}^2} \\
	&\sigma_{bW} = 500 \frac{\text{ N}}{\text{ mm}^2} \\
	&\tau_{tW} = 300 \frac{\text{ N}}{\text{ mm}^2} 
\end{align*}

\section{Freistichberechnung}
Die Sicherheit des Freistiches wird bei Umlaufbiegung und konstanter Torsions- und Zug-Druck-Beanspruchung berechnet.
\subsubsection{Abmessungen:}
\begin{align*}
	&D= 47 \text{ mm} & \frac{r}{t} = 0,21 \\
	&d= 39,4 \text{ mm} & \frac{r}{d} = 0,02 \\
	& r= 0,8 \text{ mm} & \frac{d}{D} = 0,838 \\
	& t= \frac{D-d}{2} = 3,8 \text{ mm} 
\end{align*}
\subsubsection{Wirkende Spannungen (nach DIN 743 - 1 Tabelle 1)}
\begin{align*}
	&\sigma_{zda} = \frac{F_{zda}}{A} = \frac{N}{\frac{\pi \x d^2}{4}} = \frac{581,94 \text{ N}}{1219,22 \text{ mm}^2} = 0,48 \frac{\text{ N}}{\text{ mm}^2} \\
	&\sigma_{ba} = \frac{M_{ba}}{A} = \frac{|M_{b,res}(x=345 \text{ mm}) |}{\frac{\pi \x d^3}{32}} = \frac{541240 \text{ Nmm}}{6004,66 \text{ mm}^3} = 90,1 \frac{\text{ N}}{\text{ mm}^2} \\
	&\tau_{ta} = \frac{M_{ta}}{A} = \frac{M_{\mathrm{II}}}{\frac{\pi \x d^3}{16}} = \frac{71500 \text{ Nmm}}{12009,32 \text{ mm}^3} = 5,95 \frac{\text{ N}}{\text{ mm}^2} 
\end{align*}
\subsubsection{Gesamteinflussfaktor}
Die folgenden Formeln stammen, soweit nichts anderes vermerkt, aus der DIN 743 - 2.
\begin{itemize}
	\item Bezogenes Spannungsgefälle $G'$ \hfill (Tabelle 2)
	\begin{align*}
		\text{Hilfsgröße } \Phi &= \frac{1}{4 \x \sqrt{\frac{t}{r}} +2} = 0,09 \\
		G'_{ZD} &= \frac{2,3 \x (1+\Phi)}{r} = 3,13 \frac{1}{\text{ mm}} \\
		G'_{B}&= G'_{ZD}  =3,13 \frac{1}{\text{ mm}} \\
		G'_{T}&= \frac{1,15}{r} = 1,44\frac{1}{\text{ mm}} 
	\end{align*}
	\item Technologischer Größeneinflussfaktor $K_1 (d_{eff})$ 
	\begin{align*}
		&\text{zur Vereinfachung wird angenommen } d_{eff} = D =47 \text{ mm und } d_B = 16 \text{ mm} \\ 
		&\text{Streckgrenze Vergütungsstahl: }K_{1,Re}(d_{eff}) = 1 - 0,34 \lg \left( \frac{d_{eff}}{d_B} \right) = 0,84  &(14) \\
		&\text{Zugfestigkeit Vergütungsstahl: } K_{1,Rm}(d_{eff}) = 1 - 26 \lg \left( \frac{d_{eff}}{d_B} \right) = 0,88  &(12) 
	\end{align*}
	\item Formzahl $\alpha$
	\begin{align*}
		&\alpha_{\sigma ZD} = 1 + \frac{1}{\sqrt{0,62 \x \frac{r}{t} + 7 \x \frac{r}{d} (1+2 \x \frac{r}{d})^2}} = 2,88 & \text{(Bild 8)} \\
		&\alpha_{\sigma B} = 1 + \frac{1}{\sqrt{0,62 \x \frac{r}{t} + 11,6 \x \frac{r}{d} (1+2 \x \frac{r}{d})^2 + 0,2 \x \left( \frac{r}{t} \right) ^3 \x \frac{d}{D}}} = 2,62 & \text{(Bild 9)} \\
		&\alpha_{\tau} = 1 + \frac{1}{\sqrt{3,4 \x \frac{r}{t} + 38 \x \frac{r}{d} (1+2 \x \frac{r}{d})^2 + \left( \frac{r}{t} \right) ^2 \x \frac{d}{D}}} = 1,8  & \text{(Bild 10)} 
	\end{align*}
	\item Stützziffer n \hfill (5)
	\begin{align*}
		n&= 1 + \sqrt{G' \x \text{ mm}} \x 10 ^{-\left( 0,33 + \frac{\sigma_S (d)}{712\frac{\text{ N}}{\text{ mm}^2}} \right)} \\
		\text{mit } \sigma_S (d) &= K_{1,Re} (d_{eff}) \x \sigma_S (d_B) \\
		&= 0,84 \x 800 \frac{\text{ N}}{\text{ mm}^2} = 672 \frac{\text{ N}}{\text{ mm}^2} \\
		n_{ZD} &= n_B = 1 + \sqrt{3,13} \x 10 ^{-\left( 0,33 + \frac{672}{712} \right)} \\
		&=1,09 \\
		n_{\tau} &= 1 + \sqrt{1,44} \x 10 ^{-\left( 0,33 + \frac{672}{712} \right)} \\
		&= 1,06 
	\end{align*}
	Formel für $\sigma_S$ aus Skript KL III \ccite{bib:poll:kl3} S.168
	\item Kerbwirkungszahl $\beta$ \hfill (4)
	\begin{align*}
		&\beta_{\sigma ZD} = \frac{\alpha_{\sigma ZD}}{n_{ZD}} = 2,64 \\
		&\beta_{\sigma B} = \frac{\alpha_{\sigma B}}{n_{B}} = 2,4 \\
		&\beta_{ \tau} = \frac{\alpha_{\sigma \tau}}{n_{\tau}} = 1,7 
	\end{align*}
	\item Geometrischer Größeneinflussfaktor $K_2 (d)$ 
	\begin{align*}
		&\text{Zug/Druck: }K_{2ZD} (d) = 1 & (15) \\
		&\text{Biegung/Torsion: } K_{2B,\tau} (d) = 1 -0,2 \x  \frac{\lg (d / 7,5 \text{ mm})}{lg 20}  = 0,89 &(16) 
	\end{align*}
	\item Einflussfaktor der Oberflächenrauheit $K_{F\sigma, \tau}$ (Skript KL III \ccite{bib:poll:kl3} S.171)
	\begin{align*}
		\text{Zug/Druck \& Biegung: } K_{F\sigma} &= 1 - 0,22 \x \lg \left( \frac{Rz}{\mu m}\right) \x ( \lg \left( \frac{\sigma_B (d)}{20}\frac{\text{ N}}{\text{ mm}^2}\right) -1 ) &(18) \\
		\text{mit } \sigma_B (d) &= K_{1,Re} (d_{eff}) \x \sigma_B (d_B) \\
		&= 0,84 \x 1000\frac{\text{ N}}{\text{ mm}^2} = 840 \frac{\text{ N}}{\text{ mm}^2} \\
		\implies K_{F\sigma} &= 0,89 \text{ mit } R_z = 6,3 \mu m\\
		\text{Torsion: } K_{F\tau} &= 0,575 \x K_{F\sigma} +0,425 = 0,937 & (19) 
	\end{align*}
	\item Einflussfaktor der Oberflächenverfestigung (aus KL III \ccite{bib:poll:kl3} Seite 172)
	\[
		K_{V} = 1
	\]
	\item Gesamteinflussfaktor $K_{\sigma}$ nach DIN 743 - 1
	\begin{align*}
		\text{Zug/Druck: } K_{\sigma ZD} &= \left( \frac{\beta_{\sigma ZD}}{K_{2ZD} (d)} + \frac{1}{K_{F\sigma}} -1\right) \x \frac{1}{K_V} & (8) \\
		&= \left( \frac{2,64}{1} + \frac{1}{0,89} -1\right) \x \frac{1}{1} \\
		&= 2,76 \\
		\text{Biegung: } K_{\sigma B} &= \left( \frac{\beta_{\sigma B}}{K_{2B}(d)} + \frac{1}{K_{F\sigma}} -1\right) \x \frac{1}{K_V} & (8) \\
		&= \left( \frac{2,4}{0,89} + \frac{1}{0,89} -1\right) \x \frac{1}{1} \\
		&= 2,82 \\
		\text{Torsion: } K_{\tau} &= \left( \frac{\beta_{\tau}}{K_{2\tau}(d)} + \frac{1}{K_{F\tau}} -1\right) \x \frac{1}{K_V} & (9) \\
		&= \left( \frac{1,7}{0,89} + \frac{1}{0,937} -1\right) \x \frac{1}{1} \\
		&= 1,98 
	\end{align*}
\end{itemize}
\subsubsection{Vorhandene Sicherheitszahl für Dauerfestigkeitsnachweis nach Belastungsfall 1 }
verwendete Formeln aus der DIN 743 - 1
\begin{itemize}
	\item Vergleichsmittelspannung 
	\begin{align*}
	\sigma_{mv}&= \sqrt{(\sigma_{zdm} +\sigma_{bm} )^2+ 3 \x \tau_{tm}^2 } & (23) \\
	&\sigma_{bm}= 0 \text{ (da Umlaufbiegung vorliegt)} \\
	&\sigma_{zdm} = \sigma_{zda} = 0,48\frac{\text{ N}}{\text{ mm}^2} \\
	&\tau_{tm} = \tau_{ta} = 5,95 \frac{\text{ N}}{\text{ mm}^2} \\
	\implies \sigma_{mv}&= \sqrt{(0,48 \frac{\text{ N}}{\text{ mm}^2} +0 )^2+ 3 \x (5,95 \frac{\text{ N}}{\text{ mm}^2})^2 }  \\
	&= 10,32 \frac{\text{ N}}{\text{ mm}^2} \\
	\tau_{mv} &= \frac{\sigma_{mv}}{\sqrt{3}} = 5,96 \frac{\text{ N}}{\text{ mm}^2} & (24)
	\end{align*}
	\item Bauteilwecheselfestigkeit $\sigma_{WK}$ 
	\begin{align*}
	\sigma_{zdWK}&= \frac{\sigma_{zdW} \x K_{1,Rm} (d_{eff})}{K_{\sigma ZD}}  & (5)\\
	&=  \frac{400 \frac{\text{ N}}{\text{ mm}^2}\x 0,88}{2,76} = 127,54 \frac{\text{ N}}{\text{ mm}^2}\\
	\sigma_{bWK}&= \frac{\sigma_{bW} \x K_{1,Rm} (d_{eff})}{K_{\sigma B}}  & (6 )\\
	&=  \frac{500 \frac{\text{ N}}{\text{ mm}^2}\x 0,88}{2,82} = 156 \frac{\text{ N}}{\text{ mm}^2}\\
	\tau_{tWK}&= \frac{\tau_{tW} \x K_{1,Rm} (d_{eff})}{K_{\tau}} &(7) \\
	&=  \frac{300 \frac{\text{ N}}{\text{ mm}^2}\x 0,88}{1,98} = 133,3 \frac{\text{ N}}{\text{ mm}^2}
	\end{align*}
	\item Einflussfaktor der Mittelspannungsempfindlichkeit $\Psi_{K}$ 
	\begin{align*}
	\Psi_{zd \sigma K}&= \frac{\sigma_{zdWK}}{2 \x  K_{1,Rm} (d_{eff}) \x \sigma_B (d_B) -\sigma_{zdWK}}  &(20)\\
	&=  \frac{127,54 \frac{\text{ N}}{\text{ mm}^2}}{2 \x 0,88 \x 1000\frac{\text{ N}}{\text{ mm}^2} - 127,54 \frac{\text{ N}}{\text{ mm}^2}} = 0,078 \\
	\Psi_{b \sigma K}&= \frac{\sigma_{bWK}}{2 \x  K_{1,Rm} (d_{eff}) \x \sigma_B (d_B) -\sigma_{bWK}} &(21) \\
	&=  \frac{156 \frac{\text{ N}}{\text{ mm}^2}}{2 \x 0,88 \x 1000\frac{\text{ N}}{\text{ mm}^2} - 156 \frac{\text{ N}}{\text{ mm}^2}} = 0,097 \\
	\Psi_{\tau K}&= \frac{\tau_{tWK}}{2 \x  K_{1,Rm} (d_{eff}) \x \sigma_B (d_B) -\tau_{tWK}} &(22) \\
	&=  \frac{133,3 \frac{\text{ N}}{\text{ mm}^2}}{2 \x 0,88 \x 1000\frac{\text{ N}}{\text{ mm}^2} - 133,3 \frac{\text{ N}}{\text{ mm}^2}} = 0,082 
	\end{align*}
	\item Spannungsamplitude der Bauteildauerfestigkeit
	\begin{align*}
	\sigma_{zdADK} &= \sigma_{zdWK} - \Psi_{zd \sigma K} \x \sigma_{mv} &(10) \\
	&= 127,54\frac{\text{ N}}{\text{ mm}^2} - 0,078 \x 10,32 \frac{\text{ N}}{\text{ mm}^2} = 126,74 \frac{\text{ N}}{\text{ mm}^2} \\
	\sigma_{bADK} &= \sigma_{bWK} - \Psi_{b \sigma K} \x \sigma_{mv} &(11) \\
	&= 156\frac{\text{ N}}{\text{ mm}^2} - 0,097 \x 10,32 \frac{\text{ N}}{\text{ mm}^2} = 155 \frac{\text{ N}}{\text{ mm}^2} \\
	\tau_{tADK} &= \tau_{tWK} - \Psi_{ \tau K} \x \tau_{mv} &(12) \\
	&= 133,3\frac{\text{ N}}{\text{ mm}^2} - 0,082 \x 5,96 \frac{\text{ N}}{\text{ mm}^2} = 132,8 \frac{\text{ N}}{\text{ mm}^2} 
	\end{align*}
	\item vorhandene Sicherheitszahl S 
	\begin{align*}
		&S= \frac{1}{\sqrt{\left( \frac{\sigma_{zda}}{\sigma_{zdADK}} +\frac{\sigma_{ba}}{\sigma_{bADK}} \right)^2 +\left( \frac{\tau_{ta}}{\tau_{tADK}} \right)^2 }} &(2) \\
		&S=  1,7 
	\end{align*}
\end{itemize}
\subsubsection{Vorhandene Sicherheitszahl S für Nachweis gegen Überschreiten der Fließgrenze}
verwendete Formeln aus der DIN 743 - 1
\begin{itemize}
	\item Statische Stützwirkung $K_{2F}$ nach Tabelle 3
	\begin{align*}
	&K_{2F \sigma zd} = 1 \\
	&K_{2F \sigma b} = 1,2 \\
	&K_{2F \tau} = 1,2 
	\end{align*}
	\item Erhöhungsfaktor der Fließgrenze $\gamma_{F}$ nach Tabelle 2
	\begin{align*}
	&\gamma_{F\sigma} = 1,1 \text{ (Für } \beta_{\sigma} = 2,0 \text{ bis } 3,0 \text{)} \\
	&\gamma_{F\tau} = 1 
	\end{align*}
	\item Bauteilfließgrenze
	\begin{align*}
	\sigma_{zd,bFK} &= K_{1,Re} (d_{eff}) \x K_{2F \sigma} \x \gamma_{F\sigma} \x \sigma_S (d_B) & (31) \\
	\sigma_{zdFK}&= 0,84 \x 1 \x 1,1 \x 800 \frac{\text{ N}}{\text{ mm}^2} \\
	&= 739,2 \frac{\text{ N}}{\text{ mm}^2}\\
	\sigma_{bFK}&= 0,84 \x 1,2 \x 1,1 \x 800 \frac{\text{ N}}{\text{ mm}^2} \\
	&= 887\frac{\text{ N}}{\text{ mm}^2}\\
	\tau_{tFK} &= K_{1,Re} (d_{eff}) \x K_{2F \tau} \x \gamma_{F\tau} \x \sigma_S (d_B) / \sqrt{3} &(32) \\
	&= 0,84 \x 1,2 \x 1 \x 800 \frac{\text{ N}}{\text{ mm}^2} / \sqrt{3}\\
	&= 465,58 \frac{\text{ N}}{\text{ mm}^2}
	\end{align*}
	\item Vorhandene Sicherheitszahl S \\
	Für diesen Fall wird mit den wirkenden Spannungsamplituden als maximale Spannungen gerechnet, da keine stoßartige Belastung angenommen wird. 
	\begin{align*}
	&S = \frac{1}{\sqrt{\left( \frac{\sigma_{zdmax}}{\sigma_{zdFK}}+\frac{\sigma_{bmax}}{\sigma_{bFK}} \right)^2 +\left( \frac{\tau_{tmax}}{\tau_{tFK}} \right)^2 }} & (25)\\
	&= 9,7 
	\end{align*}
\end{itemize}
Die Schwachstelle 1, also der Freistich, ist gegen Dauerbruch und plastische Verformung ausreichend ausgelegt.


\section{Berechnung der Sicherungsringnut}
Die Sicherheit der Sicherungsringnut wird bei Umlaufbiegung und konstanter Torsions- und Zug-Druck-Beanspruchung berechnet.
\subsubsection{Abmessungen:}
\begin{align*}
&D= 40 \text{ mm}  \\
&d= 37,5 \text{ mm}  \\
& m= 1,85 \text{ mm} \\
& r= 0,175 \text{ mm}\\
& t=\frac{D-d}{2} =1,25 \text{ mm} 
\end{align*}
\subsubsection{Wirkende Spannungen (nach DIN 743 - 1 Tabelle 1)}
\begin{align*}
&\sigma_{zda} = \frac{F_{zda}}{A} = \frac{N}{\frac{\pi \x d^2}{4}} = \frac{581,94 \text{ N}}{1104,47 \text{ mm}^2} = 0,53 \frac{\text{ N}}{\text{ mm}^2} \\
&\sigma_{ba} = \frac{M_{ba}}{A} = \frac{|M_{b,res}(x=345 \text{ mm}) |}{\frac{\pi \x d^3}{32}} = \frac{541240 \text{ Nmm}}{5177,19 \text{ mm}^3} = 104,5 \frac{\text{ N}}{\text{ mm}^2} \\
&\tau_{ta} = \frac{M_{ta}}{A} = \frac{M_{\mathrm{II}}}{\frac{\pi \x d^3}{16}} = \frac{71500 \text{ Nmm}}{10354,37 \text{ mm}^3} = 6,9 \frac{\text{ N}}{\text{ mm}^2} 
\end{align*}
\subsubsection{Gesamteinflussfaktor}
Die folgenden Formeln stammen, soweit nichts anderes vermerkt, aus der DIN 743 - 2. Als Bezugsdurchmesser dient $d_{BK} = 30$ mm.
\begin{itemize}
\item Bezogenes Spannungsgefälle $G'$ \hfill (Tabelle 2)
	\begin{align*}
	\text{Hilfsgröße } \Phi &= \frac{1}{4 \x \sqrt{\frac{t}{r}} +2} = 0,09 \\
	G'_{ZD} &= \frac{2,3 \x (1+\Phi)}{r} = 3,13 \frac{1}{\text{ mm}} \\
	G'_{B}&= G'_{ZD}  =3,13 \frac{1}{\text{ mm}} \\
	G'_{T}&= \frac{1,15}{r} = 1,44\frac{1}{\text{ mm}} 
	\end{align*}
\item Technologischer Größeneinflussfaktor $K_1 (d_{eff})$ 
	\begin{align*}
	&\text{zur Vereinfachung wird angenommen } d_{eff} = D =40 \text{ mm und } d_B = 16 \text{ mm} \\ 
	&\text{Streckgrenze Vergütungsstahl: }K_{1,Re}(d_{eff}) = 1 - 0,34 \lg \left( \frac{d_{eff}}{d_B} \right) = 0,86  &(14) \\
	&\text{Zugfestigkeit Vergütungsstahl: } K_{1,Rm}(d_{eff}) = 1 - 26 \lg \left( \frac{d_{eff}}{d_B} \right) = 0,897  &(12) 
	\end{align*}
\item Strukturradius $ \rho ^*$ nach Kapitel 4.2.4
	\begin{align*}
	\sigma_S (d) &= \sigma_S (d_B) \x K_{1,Re}(deff) = 800 \frac{\text{ N}}{\text{ mm}^2} \x 0,86 = 688 \frac{\text{ N}}{\text{ mm}^2} &(Bild 3)\\ 
	\rho ^* &=10 ^{-\left(0,514+0,00152\x \sigma_S (d) / \left(\frac{\text{ N}}{\text{ mm}^2} \right) \right)} \x mm= 0,028 mm
	\end{align*}
\item korrigierter Rundungsradius $ r_f$ nach Kapitel 4.2.4
	\begin{align*}
	\text{Zug/Druck, Biegung: } r_{f,ZD} =& r_{f,B} = r + 2,9 \x \rho ^* = 0,175 \text{ mm} + 2,9 \x 0,028 \text{ mm} = 0,256 \text{ mm}\\ 
	\text{Torsion: } r_{f,\tau} =& r +  \rho ^* = 0,175 \text{ mm} + 0,028 \text{ mm} = 0,2\text{ mm}
	\end{align*}
\item Hilfsgröße $ \beta^* (d_{BK})$ nach Kapitel 4.2.4
	\begin{align*}
	\text{Zug/Druck: }\beta _{\sigma,ZD} ^* (d_{BK}) =& 0,9 \x (1,27 + 1,17 \x \sqrt{t/r_{f,ZD}})  = 3,47 \\ 
	\text{Biegung: }\beta _{\sigma,B} ^* (d_{BK}) =& 0,9 \x (1,14 + 1,08 \x \sqrt{t/r_{f,B}})  = 3,174 \\ 
	\text{Torsion: }\beta _{\tau} ^* (d_{BK}) =& (1,48 + 0,45 \x \sqrt{t/r_{f,\tau}})  = 2,605
	\end{align*}
\item Kerbwirkungszahl $\beta (d_{BK}) $ nach Kapitel 4.2.4
	\begin{align*}
	\text{Für } &m/t \ge 1,4 \text{ gilt: } \beta (d_{BK}) = \beta^* (d_{BK}) \\
	\implies &\beta_{\sigma,ZD} (d_{BK}) = 3,47 \\
	&\beta_{\sigma,B}(d_{BK}) = 3,174 \\
	&\beta_{ \tau} (d_{BK})= 2,605 
	\end{align*}
\item Geometrischer Größeneinflussfaktor $K_3 (d) $ und $K_3 (d_{BK}) $ \hfill (17)
	\begin{align*}
	& K_3 (d) = 1-0,2 \x lg(\beta (d_{BK})) \x \frac{lg(d/7,5 \text{ mm})}{lg(20)}\\
	&K_{3,ZD} (d) = 1-0,2 \x lg(3,47) \x \frac{lg(37,5/7,5 \text{ mm})}{lg(20)} =0,942\\
	&K_{3,B} (d) = 1-0,2 \x lg(3,174) \x \frac{lg(37,5/7,5 \text{ mm})}{lg(20)} =0,946\\
	&K_{3,\tau} (d) = 1-0,2 \x lg(2,605) \x \frac{lg(37,5/7,5 \text{ mm})}{lg(20)} =0,955\\ \\
	&K_{3,ZD} (d_{BK}) = 1-0,2 \x lg(3,47) \x \frac{lg(30/7,5 \text{ mm})}{lg(20)} =0,95\\
	&K_{3,B} (d_{BK}) = 1-0,2 \x lg(3,174) \x \frac{lg(30/7,5 \text{ mm})}{lg(20)} =0,954\\
	&K_{3,\tau} (d_{BK}) = 1-0,2 \x lg(2,605) \x \frac{lg(30/7,5 \text{ mm})}{lg(20)} =0,962
	\end{align*}
\item Kerbwirkungszahl $\beta (d) $ \hfill (3)
	\begin{align*}
	 &\beta (d) = \beta (d_{BK}) \x \frac{K_3 (d_{BK})}{K_3 (d)} \\
	 &\beta_{\sigma,ZD} (d) =  3,47 \x \frac{0,95}{0,942} = 3,499\\
	&\beta_{\sigma,B}(d) = 3,174 \x \frac{0,954}{0,946} =3,2\\
	&\beta_{ \tau} (d)= 2,605 \x \frac{0,962}{0,955}  =2,624
	\end{align*}
\item Geometrischer Größeneinflussfaktor $K_2 (d)$ 
	\begin{align*}
	&\text{Zug/Druck: }K_{2ZD} (d) = 1 & (15) \\
	&\text{Biegung/Torsion: } K_{2B,\tau} (d) = 1 -0,2 \x  \frac{\lg (d / 7,5 \text{ mm})}{lg 20}  = 0,893 &(16) 
	\end{align*}
\item Einflussfaktor der Oberflächenverfestigung (aus KL III \ccite{bib:poll:kl3} Seite 172)
	\[
	K_{V} = 1
	\]
\item Einflussfaktor der Oberflächenrauheit $K_{F\sigma, \tau} (Rz)$
	\begin{align*}
	\text{Zug/Druck \& Biegung: } K_{F\sigma} (Rz)&= 1 - 0,22 \x \lg \left( \frac{Rz}{\mu m}\right) \x ( \lg \left( \frac{\sigma_B (d)}{20}\frac{\text{ N}}{\text{ mm}^2}\right) -1 ) &(18) \\
	\text{mit } \sigma_B (d) &= K_{1,Re} (d_{eff}) \x \sigma_B (d_B) \\
	&= 0,86 \x 1000\frac{\text{ N}}{\text{ mm}^2} = 860 \frac{\text{ N}}{\text{ mm}^2} \\
	\implies K_{F\sigma} (Rz) &= 0,889 \text{ mit } R_z = 6,3 \mu m\\
	\text{Torsion: } K_{F\tau} (Rz)&= 0,575 \x K_{F\sigma} (Rz) +0,425 = 0,936 & (19) 
	\end{align*}
\item Einflussfaktor der Oberflächenrauheit $K_{F\sigma, \tau} (Rz_B)$
	\begin{align*}
	\text{Gültig für die Probe mit der } &\text{mittleren Rauheit der Kerbe } Rz_B= 20 \mu m\\
	\text{Zug/Druck \& Biegung: } K_{F\sigma} (Rz_B)&= 1 - 0,22 \x \lg \left( \frac{Rz_B}{\mu m}\right) \x ( \lg \left( \frac{\sigma_B (d)}{20}\frac{\text{ N}}{\text{ mm}^2}\right) -1 ) &(18) \\
	\implies K_{F\sigma} (Rz_B) & = 1 - 0,22 \x \lg \left( \frac{20 \mu m}{\mu m}\right) \x ( \lg \left( \frac{860\text{ N}}{\text{ mm}^2 }{20}\frac{\text{ N}}{\text{ mm}^2}\right) -1 )\\
	&= 0,819 \\
	\text{Torsion: } K_{F\tau} (Rz_B)&= 0,575 \x K_{F\sigma} (Rz_B) +0,425 = 0,896 & (19) 
	\end{align*}
\item Einflussfaktor der Oberflächenrauheit $K_{F\sigma, \tau} $
	\begin{align*}
	\text{Zug/Druck \& Biegung: } K_{F\sigma} =&\frac{K_{F\sigma} (Rz)}{K_{F\sigma} (Rz_B)} = 1,085 &(20) \\
	\text{Torsion: } K_{F\tau} = &\frac{K_{F\tau} (Rz)}{K_{F\tau} (Rz_B)} = 1,045 &(20)
	\end{align*}
\item Gesamteinflussfaktor $K_{\sigma}$ nach DIN 743 - 1
	\begin{align*}
	\text{Zug/Druck: } K_{\sigma ZD} &= \left( \frac{\beta_{\sigma ZD}}{K_{2ZD} (d)} + \frac{1}{K_{F\sigma}} -1\right) \x \frac{1}{K_V} & (8) \\
	&= \left( \frac{3,499}{1} + \frac{1}{1,085} -1\right) \x \frac{1}{1} \\
	&= 3,42 \\
	\text{Biegung: } K_{\sigma B} &= \left( \frac{\beta_{\sigma B}}{K_{2B}(d)} + \frac{1}{K_{F\sigma}} -1\right) \x \frac{1}{K_V} & (8) \\
	&= \left( \frac{3,3}{0,893} + \frac{1}{1,085} -1\right) \x \frac{1}{1} \\
	&= 3,617 \\
	\text{Torsion: } K_{\tau} &= \left( \frac{\beta_{\tau}}{K_{2\tau}(d)} + \frac{1}{K_{F\tau}} -1\right) \x \frac{1}{K_V} & (9) \\
	&= \left( \frac{2,624}{0,893} + \frac{1}{1,045} -1\right) \x \frac{1}{1} \\
	&= 2,895 
	\end{align*}
\end{itemize}
\subsubsection{Vorhandene Sicherheitszahl für Dauerfestigkeitsnachweis nach Belastungsfall 1 }
verwendete Formeln aus der DIN 743 - 1
\begin{itemize}
	\item Vergleichsmittelspannung 
	\begin{align*}
	\sigma_{mv}&= \sqrt{(\sigma_{zdm} +\sigma_{bm} )^2+ 3 \x \tau_{tm}^2 } & (23) \\
	&\sigma_{bm}= 0 \text{ (da Umlaufbiegung vorliegt)} \\
	&\sigma_{zdm} = \sigma_{zda} = 0,53\frac{\text{ N}}{\text{ mm}^2} \\
	&\tau_{tm} = \tau_{ta} = 6,9 \frac{\text{ N}}{\text{ mm}^2} \\
	\implies \sigma_{mv}&= \sqrt{(4,22 \frac{\text{ N}}{\text{ mm}^2} +0 )^2+ 3 \x (6,9 \frac{\text{ N}}{\text{ mm}^2})^2 }  \\
	&= 11,96 \frac{\text{ N}}{\text{ mm}^2} \\
	\tau_{mv} &= \frac{\sigma_{mv}}{\sqrt{3}} = 6,9 \frac{\text{ N}}{\text{ mm}^2} & (24)
	\end{align*}
	\item Bauteilwecheselfestigkeit $\sigma_{WK}$ 
	\begin{align*}
	\sigma_{zdWK}&= \frac{\sigma_{zdW} \x K_{1,Rm} (d_{eff})}{K_{\sigma ZD}}  & (5)\\
	&=  \frac{400 \frac{\text{ N}}{\text{ mm}^2}\x 0,897}{3,42} = 104,91 \frac{\text{ N}}{\text{ mm}^2}\\
	\sigma_{bWK}&= \frac{\sigma_{bW} \x K_{1,Rm} (d_{eff})}{K_{\sigma B}}  & (6 )\\
	&=  \frac{500 \frac{\text{ N}}{\text{ mm}^2}\x 0,897}{3,617} = 124 \frac{\text{ N}}{\text{ mm}^2}\\
	\tau_{tWK}&= \frac{\tau_{tW} \x K_{1,Rm} (d_{eff})}{K_{\tau}} &(7) \\
	&=  \frac{300 \frac{\text{ N}}{\text{ mm}^2}\x 0,897}{2,895} = 92,95 \frac{\text{ N}}{\text{ mm}^2}
	\end{align*}
	\item Einflussfaktor der Mittelspannungsempfindlichkeit $\Psi_{K}$ 
	\begin{align*}
	\Psi_{zd \sigma K}&= \frac{\sigma_{zdWK}}{2 \x  K_{1,Rm} (d_{eff}) \x \sigma_B (d_B) -\sigma_{zdWK}}  &(20)\\
	&=  \frac{104,91 \frac{\text{ N}}{\text{ mm}^2}}{2 \x 0,897 \x 1000\frac{\text{ N}}{\text{ mm}^2} - 104,91 \frac{\text{ N}}{\text{ mm}^2}} = 0,062 \\
	\Psi_{b \sigma K}&= \frac{\sigma_{bWK}}{2 \x  K_{1,Rm} (d_{eff}) \x \sigma_B (d_B) -\sigma_{bWK}} &(21) \\
	&=  \frac{124 \frac{\text{ N}}{\text{ mm}^2}}{2 \x 0,897 \x 1000\frac{\text{ N}}{\text{ mm}^2} - 124 \frac{\text{ N}}{\text{ mm}^2}} = 0,074 \\
	\Psi_{\tau K}&= \frac{\tau_{tWK}}{2 \x  K_{1,Rm} (d_{eff}) \x \sigma_B (d_B) -\tau_{tWK}} &(22) \\
	&=  \frac{92,95 \frac{\text{ N}}{\text{ mm}^2}}{2 \x 0,897 \x 1000\frac{\text{ N}}{\text{ mm}^2} - 92,95 \frac{\text{ N}}{\text{ mm}^2}} = 0,055 
	\end{align*}
	\item Spannungsamplitude der Bauteildauerfestigkeit
	\begin{align*}
	\sigma_{zdADK} &= \sigma_{zdWK} - \Psi_{zd \sigma K} \x \sigma_{mv} &(10) \\
	&= 104,91\frac{\text{ N}}{\text{ mm}^2} - 0,062 \x 11,96 \frac{\text{ N}}{\text{ mm}^2} = 104,17 \frac{\text{ N}}{\text{ mm}^2} \\
	\sigma_{bADK} &= \sigma_{bWK} - \Psi_{b \sigma K} \x \sigma_{mv} &(11) \\
	&= 124\frac{\text{ N}}{\text{ mm}^2} - 0,074 \x 11,96 \frac{\text{ N}}{\text{ mm}^2} = 123,11\frac{\text{ N}}{\text{ mm}^2} \\
	\tau_{tADK} &= \tau_{tWK} - \Psi_{ \tau K} \x \tau_{mv} &(12) \\
	&= 92,95\frac{\text{ N}}{\text{ mm}^2} - 0,055 \x6,9 \frac{\text{ N}}{\text{ mm}^2} = 92,57 \frac{\text{ N}}{\text{ mm}^2} 
	\end{align*}
	\item vorhandene Sicherheitszahl S 
	\begin{align*}
	&S= \frac{1}{\sqrt{\left( \frac{\sigma_{zda}}{\sigma_{zdADK}} +\frac{\sigma_{ba}}{\sigma_{bADK}} \right)^2 +\left( \frac{\tau_{ta}}{\tau_{tADK}} \right)^2 }} &(2) \\
	&S=  1,35 
	\end{align*}
\end{itemize}
\subsubsection{Vorhandene Sicherheitszahl S für Nachweis gegen Überschreiten der Fließgrenze}
verwendete Formeln aus der DIN 743 - 1
\begin{itemize}
\item Statische Stützwirkung $K_{2F}$ nach Tabelle 3
	\begin{align*}
	&K_{2F \sigma zd} = 1 \\
	&K_{2F \sigma b} = 1,2 \\
	&K_{2F \tau} = 1,2 
	\end{align*}
\item Erhöhungsfaktor der Fließgrenze $\gamma_{F}$ nach Tabelle 2
	\begin{align*}
	&\gamma_{F\sigma} = 1,15 \text{ (Für } \beta_{\sigma} > 3,0 \text{)} \\
	&\gamma_{F\tau} = 1 
	\end{align*}
\item Bauteilfließgrenze
	\begin{align*}
	\sigma_{zd,bFK} &= K_{1,Re} (d_{eff}) \x K_{2F \sigma} \x \gamma_{F\sigma} \x \sigma_S (d_B) & (31) \\
	\sigma_{zdFK}&= 0,86 \x 1 \x 1,15 \x 800 \frac{\text{ N}}{\text{ mm}^2} \\
	&= 791,2 \frac{\text{ N}}{\text{ mm}^2}\\
	\sigma_{bFK}&= 0,86 \x 1,2 \x 1,15 \x 800 \frac{\text{ N}}{\text{ mm}^2} \\
	&= 949,44\frac{\text{ N}}{\text{ mm}^2}\\
	\tau_{tFK} &= K_{1,Re} (d_{eff}) \x K_{2F \tau} \x \gamma_{F\tau} \x \sigma_S (d_B) / \sqrt{3} &(32) \\
	&= 0,86 \x 1,2 \x 1 \x 800 \frac{\text{ N}}{\text{ mm}^2} / \sqrt{3}\\
	&= 476,66 \frac{\text{ N}}{\text{ mm}^2}
	\end{align*}
	\newpage
\item Vorhandene Sicherheitszahl S \\
	Für diesen Fall wird mit den wirkenden Spannungsamplituden als maximale Spannungen gerechnet, da keine stoßartige Belastung angenommen wird. 
	\begin{align*}
	&S = \frac{1}{\sqrt{\left( \frac{\sigma_{zdmax}}{\sigma_{zdFK}}+\frac{\sigma_{bmax}}{\sigma_{bFK}} \right)^2 +\left( \frac{\tau_{tmax}}{\tau_{tFK}} \right)^2 }} & (25)\\
	&= 8,95 
	\end{align*}
\end{itemize}
Die Schwachstelle 2, also die Sicherungsringnut, ist ebenfalls gegen Dauerbruch und plastische Verformung ausreichend ausgelegt, da er die geforderte Sicherheit von 1,2 erfüllt.