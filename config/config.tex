% config
\documentclass[parskip=half,ngerman,paper=a4,abstract,fontsize=11pt,numbers=noendperiod,bibliography=totocnumbered,listof=totoc]{scrreprt}
\usepackage{scrpage2}
\automark[chapter]{section}

% pdf
\pdfminorversion=4

% Fussnoten in Tabellen
\usepackage{threeparttable}

% table of content configuration
\setcounter{secnumdepth}{2}
\setcounter{tocdepth}{2}

% heading for appendix
\clearscrheadfoot
\cfoot{\pagemark}
\ohead{\textnormal{\textbf{\headmark}}}


\ohead{\small{Jennifer Wozniak, 10004257}}
%\usepackage{scrpage2}
\pagestyle{scrheadings}





% layout
\usepackage[left=4cm,right=2cm,top=1.5cm,bottom=2.5cm,includeheadfoot,headheight=1.3\baselineskip]{geometry}
\renewcommand{\chapterpagestyle}{scrheadings}
\renewcommand*{\chapterheadstartvskip}{\vspace*{-0.5\baselineskip}}
% footnote
\usepackage{remreset}
\makeatletter\@removefromreset{footnote}{chapter}\makeatother

\usepackage{lmodern}
\usepackage{babel}
\usepackage[utf8]{inputenc}
\usepackage[babel,german=quotes]{csquotes}

% 1,5 line space
\usepackage{setspace}
\onehalfspacing

% landscape sites
\usepackage{lscape}

% pictures
\usepackage{graphicx}
\newcommand{\includegraphix}[2]{\IfFileExists{#1}{\includegraphics[#2]{#1}}{\includegraphics[draft,#2]{#1}}}
\usepackage{placeins}
\usepackage{wrapfig}
\usepackage{rotating}

% bibliography
\usepackage[style=alphabetic-verb,block=space,backend=biber,bibencoding=utf8]{biblatex}
\bibliography{library/bibliography}

\defbibheading{Literatur}{\section{Literatur}}
\defbibheading{Internet}{\section{Internet}}

% links
\usepackage[pdfborder={0 0 0}, unicode=true]{hyperref}
\usepackage{pdfsync}

% floatingareas
\let\mysection=\section
\renewcommand{\section}[1]{\FloatBarrier\mysection{#1}}

\let\mysubsection=\subsection
\renewcommand{\subsection}[1]{\FloatBarrier\mysubsection{#1}}

\let\mysubsubsection=\subsubsection
\renewcommand{\subsubsection}[1]{\FloatBarrier\mysubsubsection{#1}}

\let\myparagraph=\paragraph
\renewcommand{\paragraph}[1]{\FloatBarrier\myparagraph{#1}}

\let\mysubparagraph=\subparagraph
\renewcommand{\subparagraph}[1]{\FloatBarrier\mysubparagraph{#1}}

\usepackage{float}

% glossaries
\usepackage[acronym,nopostdot]{glossaries}
\renewcommand{\acronymname}{Abkürzungsverzeichnis}
\renewcommand*{\glssettoctitle}[1]{\def\glossarytoctitle{\csname @glotype@#1@title\endcsname}}

\let\mygls=\gls
\renewcommand{\gls}[1]{\protect\glsIfListOfAcronyms{\glsentrytype{#1}}{\textnormal{\textit{\mygls{#1}}}}{\textnormal{\textbf{\mygls{#1}\footnote{siehe Glossar: \mygls{#1}}}}}}
\newcommand{\glsLink}[1]{\mygls{#1}}

\let\myglspl=\glspl
\renewcommand{\glspl}[1]{\protect\glsIfListOfAcronyms{\glsentrytype{#1}}{\textnormal{\textit{\myglspl{#1}}}}{\textnormal{\textbf{\myglspl{#1}\footnote{siehe Glossar: \mygls{#1}}}}}}

\let\myglsfirst=\glsfirst
\renewcommand{\glsfirst}[1]{\protect\glsIfListOfAcronyms{\glsentrytype{#1}}{\textnormal{\textit{\myglsfirst{#1}}}}{\textnormal{\textbf{\myglsfirst{#1}\footnote{siehe Glossar: \mygls{#1}}}}}}

\let\myglsdisp=\glsdisp
\renewcommand{\glsdisp}[2]{\protect\glsIfListOfAcronyms{\glsentrytype{#1}}{\textnormal{\textit{\myglsdisp{#1}{#2}}}}{\textnormal{\textbf{\myglsdisp{#1}{#2}\footnote{siehe Glossar: \mygls{#1}}}}}}

\newglossarystyle{myListStyle}{\setglossarystyle{list}\renewcommand*{\glossaryentryfield}[5]{\item[\glstarget{##1}{##2}]##3\glspostdescription\dotfill\space{##5}}}

\makeatletter
\renewcommand*\dotfill{\leavevmode%
	\leaders\hbox{$\m@th \mkern \@dotsep mu\hbox{.}\mkern \@dotsep mu$}\hfill\kern\z@}
\makeatother

% todos
\usepackage{todonotes}
\reversemarginpar

\usepackage{color}
\definecolor{todo}{rgb}{1,.7,.6}
\definecolor{comment}{rgb}{1,1,.8}
\definecolor{correction}{rgb}{.7,1,.7}

\let\mytodo=\todo
\renewcommand{\todo}[1]{\protect\mytodo[fancyline, backgroundcolor=todo, size=\footnotesize]{TODO:\\#1}}
\newcommand{\TODO}[1]{\protect\mytodo[fancyline, backgroundcolor=todo, size=\footnotesize]{TODO:\\#1}}
\newcommand{\comment}[1]{\protect\mytodo[fancyline, backgroundcolor=comment, size=\footnotesize]{Comment:\\#1}}
\newcommand{\correction}[1]{\protect\mytodo[fancyline, backgroundcolor=correction, size=\footnotesize]{Correction:\\#1}}
 
% zitate
\let\mycite = \cite
\renewcommand{\cite}[1]{\footnote{\mycite{#1}}}
\newcommand{\ccite}[1]{\footnote{Vgl. \mycite{#1}}}

% Grafiken auf der Titelseite
\usepackage{subfigure}
\usepackage[font=small]{caption} % Bei zu wenig Text das hier wieder rausnehmen

%listings
\usepackage{listings}
%\renewcommand{\lstlistingname}{Beispiel}
\lstset{breaklines=true,captionpos=b,title=\lstname}

%custom enumerate 
\usepackage{enumitem}

%math symbols
\usepackage {amsmath}
\usepackage{amssymb}
\usepackage{amstext}
\usepackage{amsfonts}
\usepackage{mathrsfs}
\usepackage{nicefrac}
\usepackage{amsthm}
\newtheorem{mydef}{Definition}
\newtheorem{example}{Beispiel}
%Eigene Befehle:
\newcommand{\x}{\cdot}
%flowchart
\usepackage{tikz}
\usetikzlibrary{shapes.geometric, arrows}
\tikzstyle{startstop} = [rectangle, rounded corners, minimum width=3cm, minimum height=1cm, text centered, draw=black, fill=red!30]
\tikzstyle{io} = [trapezium, trapezium left angle=70, trapezium right angle=110, text width=5.2cm, minimum height=1cm, text centered, draw=black, fill=blue!30]
\tikzstyle{process} = [rectangle,text width=6cm, minimum height=1cm, text centered, draw=black, fill=orange!30]
\tikzstyle{decision} = [diamond, minimum width=3cm, minimum height=1cm, text centered, draw=black, fill=green!30]
\tikzstyle{arrow} = [thick,->,>=stealth]

%url
\usepackage{url}

% MakeIndex NICHT NÖTIG
%\usepackage{makeidx}
\makeindex